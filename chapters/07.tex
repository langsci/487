\documentclass[output=paper,colorlinks,citecolor=brown]{langscibook}
\ChapterDOI{10.5281/zenodo.17158186}
\author{Yurie Hara\orcid{0000-0002-9122-352X}\affiliation{Hokkaido University} and Mengxi Yuan\orcid{0000-0001-7667-2493}\affiliation{Jinan University}}
\title{Contextual bias and the landscape of Mandarin polar questions}
\abstract{Mandarin Chinese has at least three kinds of polar questions,  positive \textit{ma} questions, negative \textit{ma} questions  and A-not-A questions, which have similar semantics but are not totally interchangeable.  In particular, they carry different bias connotations. We show that their bias meanings and distribution are best characterized by the notion of contextual bias, which is formalized in terms of the subjective probability distribution within the framework of \quotecite{FB} Table model.  Our analysis offers a simple lexical semantics for \textit{ma} questions by employing pragmatic competition, and supports the idea that prosodic contours like the Final Fall in A-not-A questions are intonational morphemes that carry semantic contents.}
  
%move the following commands to the "local..." files of the master project when integrating this chapter

\IfFileExists{../localcommands.tex}{%hack to check whether this is being compiled as part of a collection or standalone
   \usepackage{tabularx,multicol}
%	\setlength{\multicolsep}{6.0pt plus 2.0pt minus 2.0pt}
\usepackage{array} % for the 'm' column type
\usepackage{multirow}

%font:
\usepackage{siunitx}
\sisetup{group-digits=none}

\usepackage{textcomp} %emdash

%\usepackage{libertinus−otf}
%\setmainfont{Libertinus}

 
\usepackage{langsci-optional}
\usepackage{langsci-lgr}
\usepackage{langsci-gb4e}
\usepackage{langsci-basic}
\usepackage{langsci-affiliations}
\usepackage{langsci-branding}

\usepackage{url}
\urlstyle{same}
\usepackage{orcidlink}

%\usepackage{langsci-textipa}

\usepackage{amsmath}
\usepackage{amssymb}
\usepackage{stmaryrd}

%\usepackage{biblatex}%citation input! DO NOT CHANGE!
%\usepackage[american]{babel}
%\usepackage{csquotes}
%\usepackage[style=apa]{biblatex}
%\bibliographystyle{linquiry2}
%\usepackage[hidelinks, bookmarks=false, pdfstartview=FitH]{hyperref} %bookmarks=false, urlcolor=blue,

\usepackage{pifont} %checkmarks
%\usepackage{ulem}

%no new packages for 02.tex :

%\usepackage{setspace}
%\doublespacing
%\singlespacing


\usepackage{tikz-qtree}
\usepackage{tikz-qtree-compat}
\tikzset{every tree node/.style={align=center,anchor=north}}
%\usepackage{qtree,tree-dvips}%for trees, dvips won't work with figures unless figures are converted to .eps. make sure Typeset is set to "tex and DVI", not "pdftex" 
%\qtreecentertrue
\usepackage[linguistics]{forest}
\forestset{
fairly nice empty nodes/.style={
delay={where content={}
{shape=coordinate, for siblings={anchor=north}}{}},
for tree={s sep=4mm} }
            }


\usepackage[nameinlink]{cleveref} %for \Cref{}
% \usepackage{comment}
% \usepackage{color}
% \usepackage{subcaption}
\usepackage{subfigure}
% \usepackage{caption}
\usepackage{arydshln}

% \usepackage[scale=0.8]{FiraMono}

%\usepackage[export]{adjustbox}

%03.tex packages:
%\usepackage{linguex}%for examples

%04.tex packages:
\usepackage{cancel}
\usepackage{metre}
% \pagenumbering{roman}

%08.tex packages:
% \usepackage{xeCJK} %Chinese fonts
% \newfontfamily{\NotoSerifTC}{Noto Serif TC}
% \setCJKmainfont{Noto Serif TC}

   %for all .tex files (Affiliation setups):
\SetupAffiliations{ output in groups = false,
separator between two = {\bigskip\\},
separator between multiple = {\bigskip\\},
separator between final two = {\bigskip\\},
orcid placement=after
}

% ORCIDs in langsci-affiliations 
\definecolor{orcidlogocol}{cmyk}{0,0,0,1}
\RenewDocumentCommand{\LinkToORCIDinAffiliations}{ +m }
  {%
    \,\orcidlink{#1}%
  }

\makeatletter
\let\thetitle\@title
\let\theauthor\@author
\makeatother

\newcommand{\togglepaper}[1][0]{
   \bibliography{../localbibliography}
   \papernote{\scriptsize\normalfont
     \theauthor.
     \titleTemp.
     To appear in:
     E. Di Tor \& Herr Rausgeberin (ed.).
     Booktitle in localcommands.tex.
     Berlin: Language Science Press. [preliminary page numbering]
   }
   \pagenumbering{roman}
   \setcounter{chapter}{#1}
   \addtocounter{chapter}{-1}
}

\newbool{bookcompile}
\booltrue{bookcompile}
\newcommand{\bookorchapter}[2]{\ifbool{bookcompile}{#1}{#2}}

\newcommand{\cmark}{\ding{51}}%
\newcommand{\xmark}{\ding{55}}%

%for 01.tex:
\newcommand{\eval}[2]{\llbracket #1\rrbracket^{#2}}

%02.tex:
\newcommand{\smiley}{:)}

%03.tex:
\newcommand{\exa}{\ea}
%\renewcommand{\firstrefdash}{}%changes citations from, e.g., (2-a) to (2a)
\newcommand{\den}[2]{\ensuremath{\llbracket#1\rrbracket}\textsuperscript{\ensuremath{#2}}}
\newcommand{\citepos}[1]{\citeauthor{#1}'s (\citeyear{#1})}
\newcommand{\tp}[1]{\ensuremath{{\langle #1 \rangle}}}

\newcommand{\rise}{$\nearrow$\xspace}
\newcommand{\fall}{$\searrow$\xspace}
\newcommand{\notp}{\emph{not}-$p$\xspace}

%09.tex:
%\newcommand\den[1]{\ensuremath{[\![ #1 ]\!]}}
%\newcommand{\int}[1]{\ensuremath{\llbracket #1 \rrbracket}}

%13.tex:
\newcommand{\PreserveBackslash}[1]{\let\temp=\\#1\let\\=\temp}
% \newcolumntype{C}[1]{>{\PreserveBackslash\centering}p{#1}}
\newcolumntype{R}[1]{>{\PreserveBackslash\raggedleft}p{#1}}
\newcolumntype{L}[1]{>{\PreserveBackslash\raggedright}p{#1}}

\newcommand{\SB}{\textsubscript}
\newcommand{\SuB}{\textsuperscript}

\newcommand{\quotecite}[1]{\citeauthor{#1}'s (\citeyear*{#1})}

   %% hyphenation points for line breaks
%% Normally, automatic hyphenation in LaTeX is very good
%% If a word is mis-hyphenated, add it to this file
%%
%% add information to TeX file before \begin{document} with:
%% %% hyphenation points for line breaks
%% Normally, automatic hyphenation in LaTeX is very good
%% If a word is mis-hyphenated, add it to this file
%%
%% add information to TeX file before \begin{document} with:
%% %% hyphenation points for line breaks
%% Normally, automatic hyphenation in LaTeX is very good
%% If a word is mis-hyphenated, add it to this file
%%
%% add information to TeX file before \begin{document} with:
%% \include{localhyphenation}
\hyphenation{
    par-a-digm
}
\hyphenation{
que-stions
}
\hyphenation{
na-me-l-y
}
\hyphenation{
ge-ne-ra-tion
}
\hyphenation{
Hir-sch-berg}
\hyphenation{
stee-p-er
}
\hyphenation{
inter-ro-ga-tives
}
\hyphenation{
cons-truc-tion
}
\hyphenation{
p-u-sh-ed
}
\hyphenation{
A-mong
}
\hyphenation{
award-ed
}
\hyphenation{
synta-ctic
}
%\hyphenation{
%wh-ich
%}
\hyphenation{
call-ed
}
\hyphenation{
mo-no-po-lar
}
\hyphenation{
proso-dic
}
\hyphenation{
non-ve-ri-di-cal
}
\hyphenation{
Ro-me-ro
}
\hyphenation{
though
}
\hyphenation{
ra-ther
}
\hyphenation{
mo-da-li-ty
}
\hyphenation{
prag-ma-ti-cal-ly
}
\hyphenation{
trans-ver-sal
}
\hyphenation{
re-se-arch
}
\hyphenation{
clau-s-es
}
\hyphenation{
c-lau-se
}
\hyphenation{
spea-k-er
}
\hyphenation{
a-mon-g-st
}
\hyphenation{
th-rou-gh
}
\hyphenation{
ad-dres-see
}
\hyphenation{
mo-da-li-s-ed
}
\hyphenation{
Ja-mie-son}








\hyphenation{
    par-a-digm
}
\hyphenation{
que-stions
}
\hyphenation{
na-me-l-y
}
\hyphenation{
ge-ne-ra-tion
}
\hyphenation{
Hir-sch-berg}
\hyphenation{
stee-p-er
}
\hyphenation{
inter-ro-ga-tives
}
\hyphenation{
cons-truc-tion
}
\hyphenation{
p-u-sh-ed
}
\hyphenation{
A-mong
}
\hyphenation{
award-ed
}
\hyphenation{
synta-ctic
}
%\hyphenation{
%wh-ich
%}
\hyphenation{
call-ed
}
\hyphenation{
mo-no-po-lar
}
\hyphenation{
proso-dic
}
\hyphenation{
non-ve-ri-di-cal
}
\hyphenation{
Ro-me-ro
}
\hyphenation{
though
}
\hyphenation{
ra-ther
}
\hyphenation{
mo-da-li-ty
}
\hyphenation{
prag-ma-ti-cal-ly
}
\hyphenation{
trans-ver-sal
}
\hyphenation{
re-se-arch
}
\hyphenation{
clau-s-es
}
\hyphenation{
c-lau-se
}
\hyphenation{
spea-k-er
}
\hyphenation{
a-mon-g-st
}
\hyphenation{
th-rou-gh
}
\hyphenation{
ad-dres-see
}
\hyphenation{
mo-da-li-s-ed
}
\hyphenation{
Ja-mie-son}








\hyphenation{
    par-a-digm
}
\hyphenation{
que-stions
}
\hyphenation{
na-me-l-y
}
\hyphenation{
ge-ne-ra-tion
}
\hyphenation{
Hir-sch-berg}
\hyphenation{
stee-p-er
}
\hyphenation{
inter-ro-ga-tives
}
\hyphenation{
cons-truc-tion
}
\hyphenation{
p-u-sh-ed
}
\hyphenation{
A-mong
}
\hyphenation{
award-ed
}
\hyphenation{
synta-ctic
}
%\hyphenation{
%wh-ich
%}
\hyphenation{
call-ed
}
\hyphenation{
mo-no-po-lar
}
\hyphenation{
proso-dic
}
\hyphenation{
non-ve-ri-di-cal
}
\hyphenation{
Ro-me-ro
}
\hyphenation{
though
}
\hyphenation{
ra-ther
}
\hyphenation{
mo-da-li-ty
}
\hyphenation{
prag-ma-ti-cal-ly
}
\hyphenation{
trans-ver-sal
}
\hyphenation{
re-se-arch
}
\hyphenation{
clau-s-es
}
\hyphenation{
c-lau-se
}
\hyphenation{
spea-k-er
}
\hyphenation{
a-mon-g-st
}
\hyphenation{
th-rou-gh
}
\hyphenation{
ad-dres-see
}
\hyphenation{
mo-da-li-s-ed
}
\hyphenation{
Ja-mie-son}








    \bibliography{localbibliography}
    \togglepaper[23]
}{}

\begin{document}
\maketitle

\section{Introduction}

According to \citeauthor{Hamblin5}'s (\citeyear{Hamblin5}, \citeyear{Hamblin}) seminal theory of questions,  a question denotes a  set of propositions that count as possible answers to it, which predicts that there is no semantic difference between English positive polar
questions, negative polar questions and  alternative questions of the form `$p$ or not $p$?'. However, a number of subsequent studies present evidence   that these  questions have different properties and are not always interchangeable (\citealt{Bolinger, Ladd1981, BG, vs, Romero, Biezma, Roelofsen2010, BR, Sudo, Krifka15, Domaneschi, FR2017}, a.o.).

Mandarin Chinese also has  three  types of  polar questions that are not interchangeable: positive \textit{ma} questions  (+\textsc{maq}), negative \textit{ma} questions  ($-$\textsc{maq}) and A-not-A questions (\textsc{anaq}).  \emph{Ma}-questions, henceforth \textsc{maq}s, are formed by attaching a question particle \emph{ma} at the end of the sentence as in \xxref{mapos}{maneg}.  As we will see below, positive and negative \emph{ma} questions are in complementary distribution.

\ea\label{mapos}
\gll Xia yu le ma?\\
	fall rain  \textsc{perf} \textsc{q}\\
\glt	`Did it rain?' \jambox*{(+\textsc{maq})} 
	\z

	
	
	
\ea\label{maneg}
\gll Mei xia yu  ma?\\
not fall rain   \textsc{q}\\
\glt 	`Did it not rain?' \jambox*{($-$\textsc{maq})} 
\z


 A-not-A questions, henceforth \textsc{anaq}s,  conjoin the verb and its negative counterpart and end with an obligatory final boundary low tone L\% (henceforth the final Low tone is indicated by as `$\downarrow$')	and an optional question particle \emph{ne}:
	
\ea \label{anaq}
\gll Xia mei xia yu (ne)?$\downarrow$/L\%\\
	fall not fall rain \textsc{q}\\
\glt	`Did it rain or not rain?'  \jambox*{(\textsc{anaq})} 
	\z


While intuitively \xxref{mapos}{anaq} are  asking the same question of whether it rained, they  are used  in different contexts.  This naturally raises the following question: What determines the distribution of the three kinds of questions?  \citet{YuanHaraGlowinAsia2019} discuss the syntactic and prosodic differences among the three constructions as well as the differences in compositional semantics.  The current paper focuses on the bias profile/type of bias that arise from these constructions.



\section{Data}\label{sec:data}
This section presents the main empirical observation regarding Mandarin polar questions. In particular, we show which constructions are available in what kind of contexts.


\subsection{Positive \textit{ma} question}

Let us first take a look at positive \textit{ma} questions (+\textsc{maq}).  A +\textsc{maq} can be used in out-of-blue contexts where no conversation participants have expressed any bias:
	

	
\begin{exe}
\ex \label{neutral} A researcher uses a questionnaire to investigate the relationship between the weather and people's mental states. The first question in the questionnaire is:\\
	\gll  Ni de chengshi zuotian  xia yu le ma?\\
you \textsc{gen} city  yesterday fall rain \textsc{perf} ma\\
\glt `Did it rain yesterday in your city?' \jambox*{(+\textsc{maq})}

\end{exe}
	
It can also be used in positively biased contexts.  In \xref{eg:pbma}, Speaker A's assertion of $p$ renders the context biased towards $p$. 

\begin{exe}
\ex\label{eg:pbma}
A: \gll Zuowan (henkeneng) xia yu le.\\
		 last-night probably fall rain \textsc{perf}\\
\glt `(Probably,) It rained last night.'
\sn B: \gll Xia yu le ma? \\
    fall rain \textsc{perf} ma\\
\glt `Did it rain?' \jambox*{(+\textsc{maq})}
\end{exe}


+\textsc{maq} are also felicitous in contexts that are non-verbally biased towards p:


\ea\label{wetma}
B enters A's windowless room wearing a dripping wet raincoat.
\sn A: \gll Xia yu le ma? \\
fall rain \textsc{perf} ma\\
\glt `Did it rain?'  \jambox*{(+\textsc{maq})}
\z


In contrast, when the context is biased toward $\neg p$, a  +\textsc{maq} cannot be used:

\ea\label{1qing} A and B open the window and find the ground dry. A speaks to B, who stayed up all night.
\sn A: \#\gll Zuowan xia yu le ma?\\ 
last-night fall rain \textsc{perf} ma\\
\glt `Did it rain last night?'\jambox*{(+\textsc{maq})} 
\z

As summarized in \tabref{matab}, +\textsc{maq}s are felicitous when the context is neutral or positively biased.

\begin{table}
\begin{tabularx}{\textwidth}{lCCc}
\lsptoprule
&  Neutral  & biased towards  $p$ & biased towards $\neg p$\\
\midrule
positive \textsc{maq}s & \cmark & \cmark & \#\\
\lspbottomrule
\end{tabularx}
\caption{Distribution of positive \textsc{maq}s}
\label{matab}
\end{table}
	
\subsection{Negative \textit{ma} questions}
	
Negative \textit{ma} questions ($-$\textsc{maq}s)  are in 	complementary distribution with +\textsc{maq}s.  Thus, they cannot be used in an out-of-the-blue context like \xref{dii}.

\begin{exe}
\ex \label{dii} A researcher uses a questionnaire to investigate the relationship between the weather and people's mental states. The first question in the questionnaire is:
\sn[\#]{
\gll Ni de chengshi zuotian mei xia yu  ma?\\
you \textsc{gen} city  yesterday \textsc{neg} fall rain  ma\\}
\glt 	`Did it not rain  yesterday in your city?' \jambox*{($-$\textsc{maq})}
\end{exe}
	
Negative \textsc{maq}s are also disallowed in positively biased contexts as in \xxref{posnmaq}{nonv}.

\begin{exe}
\ex \label{posnmaq} 
\begin{xlist}
\exi{A:}
\gll Zuowan (henkeneng) xia yu le.\\
last-night probably fall rain \textsc{perf}\\
\glt  `It (probably) rained last night.' 
\exi{B:}\#\gll Mei xia yu ma?\\
\textsc{neg} fall rain ma\\
\glt `Did it not rain?'\jambox*{($-$\textsc{maq})}
\end{xlist}
\ex \label{nonv} B enters A's windowless room wearing a dripping wet raincoat.
\begin{xlist}
\exi{A:} \#\gll Mei xia yu ma?\\
\textsc{neg} fall rain ma\\
\glt `Did it not rain?'\jambox*{($-$\textsc{maq})}
\end{xlist}
\end{exe}
	
	
This raises the questions of when -MAQs can be used. The answer is that they can be used in contexts that exclude +MAQs when the context is biased towards $\neg p$.  The bias can arise verbally  as in \xref{xiayune} or non-verbally as in \xref{nonvnp}.
	
	
\ea\label{xiayune}
\begin{xlist}
\exi{A:} \gll Zuowan mei xia yu. \\
last-night \textsc{neg} fall rain\\
\glt `It did not rain last night.'
\exi{B:} \gll Mei xia yu  ma? \\
\textsc{neg} fall rain ma\\
\glt `Did it not rain?' \jambox*{($-$\textsc{maq})}
\end{xlist}
\z


\begin{exe}
\ex\label{nonvnp}  B leaves A's  windowless  room carrying a raincoat. When B returns, A notices that B's raincoat is dry.
\begin{xlist}
\exi{A:} \gll Mei xia yu  ma? \\
\textsc{neg} fall rain ma\\
\glt `Did it not rain?'\jambox*{($-$\textsc{maq})}
\end{xlist}
\end{exe}

\tabref{matable1} summarizes the distribution of \textsc{maq}s.  Positive and negative \textsc{maq}s are in complementary distribution.  Negative \textsc{maq}s are uttered when the context is biased towards $\neg p$ while positive \textsc{maq}s  are uttered elsewhere, i.e., when the context is neutral or biased towards $p$.\footnote{In Mandarin, there is another type of polar questions that contains a negation morpheme \emph{bu-shi}.  
\emph{Bu-shi} questions, like English inner high negation questions, are used when the speaker has a prior bias toward the positive answer but the utterance context is biased toward the negative one as in \xref{married}. This example was suggested by the reviewer, and the reviewer pointed out that this example is a high negative polar question. We clarify that this paper only discusses the semantics of unmarked negative \textit{ma} questions which are comparable to low negative polar questions in English, and we don't discuss the semantics of \textit{bu-shi} questions (which are comparable to high negative polar questions in English).

\begin{exe}
\ex\label{married}  B told A that he was married. On the next day, A found B at a bachelor party.
 \begin{xlist}
  \exi{A:} \gll ni bu-shi jiehun-le ma?  \\
 you not-\textsc{shi} married-\textsc{asp} \textsc{q}\\
 \glt `Aren't you married?'
\end{xlist}
\end{exe}
 

\textit{bu-shi} questions are distinct from (negative) \textsc{maq}s. \textit{Bu-shi} questions are comparable to high negative polar questions in English and unmarked negative \textsc{maq}s are comparable to low negative polar questions in English. See \citet{Fu2021} for an analysis that uses \quotecite{Romero} VERUM operator.}

\begin{table}
  \begin{tabularx}{\textwidth}{lCcc}
  \lsptoprule
  	&  Neutral  & biased towards  $p$ & biased towards $\neg p$\\
\midrule
	positive (+) \textsc{maq}s & \cmark & \cmark & \#\\
	negative (-) \textsc{maq}s & \# & \# & \cmark\\
\lspbottomrule
\end{tabularx}
\caption{Distribution of \textsc{maq}s}
\label{matable1}
\end{table}
	\subsection{A-not-A question}
	
	A-not-A questions are used only in neutral contexts as in \xref{neutcon}.  
	
\begin{exe}
\ex\label{neutcon}  A researcher uses a questionnaire to investigate the relationship between the weather and people's mental states. The first question in the questionnaire is:
\begin{xlist}
\exi{A:}
\gll Ni de chengshi zuotian  xia mei xia yu?$\downarrow$\\
	you \textsc{gen} city yesterday fall not fall rain\\
\glt	`Did it rain or not rain  yesterday in your city?'\jambox*{(\textsc{anaq})}
\end{xlist}
\end{exe}	

Once the context is biased towards either answer, questioning with an \textsc{anaq} becomes infelicitous.  In \xxref{posca}{posanonv}, the context is positively biased and the \textsc{anaq} is ruled out:
	
\begin{exe}
\ex\label{posca}
\begin{xlist}
\exi{A:}\gll Zuowan  xia yu le.\\
	last-night  fall rain \textsc{perf}\\
\glt `It rained last night.'
\exi{B:}\#\gll
	Xia mei xia yu?$\downarrow$\\
    fall \textsc{neg} fall rain\\
\glt `Did it rain or not rain?' \jambox*{(\textsc{anaq})}
\end{xlist}
\end{exe}		
		

\begin{exe}
\ex\label{posanonv}  B enters A's windowless room wearing a dripping wet raincoat.
\begin{xlist}
\exi{A:} \#\gll
Xia mei xia yu?$\downarrow$\\
fall \textsc{neg} fall rain\\
\glt `Did it rain or not rain?' \jambox*{(\textsc{anaq})}
	\end{xlist}
\end{exe}
	

	Likewise, in \xxref{negnonv2}{1qing2}, the context is negatively biased and the \textsc{anaq} is infelicitous.
	
	\begin{exe}
	\ex\label{negnonv2}
\begin{xlist}
\exi{A: }\gll Zuowan mei xia yu. \\
	last-night \textsc{neg} fall rain \\
\glt `It did not rain last night.'
\exi{B:}\#\gll
	Xia mei xia yu?$\downarrow$\\
    fall \textsc{neg} fall rain\\
\glt `Did it rain or not rain?'  \jambox*{(\textsc{anaq})}
\end{xlist}
\ex\label{1qing2} B leaves A's  windowless  room carrying a raincoat. When B returns, A notices that B's raincoat is dry.
\begin{xlist}
\exi{A:}\#\gll
	Xia mei xia yu?$\downarrow$\\
    fall \textsc{neg} fall rain\\
\glt `Did it rain or not rain?' \jambox*{(\textsc{anaq})}
\end{xlist}
\end{exe}
	
\tabref{tab:maq:anaq} summarizes the distribution of $+/-$ \textsc{maq}s and 	\textsc{anaq}s.

\begin{table}
\begin{tabularx}{\textwidth}{lCCc}
\lsptoprule
	&  neutral  & biased towards  p & biased towards $\neg p$\\
	\midrule
	positive \textsc{maq}s & \cmark & \cmark & \#\\
	negative \textsc{maq}s & \# & \# & \cmark\\
	\textsc{anaq}s & \cmark & \# & \#\\
	\lspbottomrule
\end{tabularx}
\caption{Distribution of \textsc{maq}s and 	\textsc{anaq}s} 
\label{tab:maq:anaq}
\end{table}	

The next question we will address is what kind of bias is involved. Put another way, what exactly does it mean to say `the context is biased/neutral'?




\section{Question bias in traditional grammar}


Traditional grammarians attempted to analyze  the meanings of \textsc{maq}s and \textsc{anaq}s with regard to the speaker's bias.
On the one hand, most traditional linguists  conclude that  the speaker of a \textsc{anaq} is neutral between a positive and a negative answer. On the other hand,  the nature  of the bias expressed by \textsc{maq}s  was controversial.   According to \citet[168]{Wangli1943}, for instance,   \textsc{maq}s are confirmation-seeking questions that encode the speaker's bias towards the prejacent proposition $p$, whereas \citet[356]{Chaospoken1968} and \citet[72]{Shao} claim that  \textsc{maq}s signify the speaker's bias towards the negation of the    prejacent proposition $\neg p$.  

We regard these   traditional approaches as problematic in several respects.  First, the semantics of  \textsc{maq}s and \textsc{anaq}s and their biases   are not compositionally derived  but stipulated (see \citealt{YuanHaraGlowinAsia2019} for compoisitional semantics of these questions). Second, empirical data shows that the bias is not lexically encoded in \textsc{maq}s. To illustrate, the  speaker of the +\textsc{maq} in \xref{wetma} seems to be biased towards the positive answer while the speaker in \xref{neutral} seems to be neutral.  As shown in \xref{xiayu}, furthermore, the same +\textsc{maq} can be uttered in  various contexts with different speaker biases. If a +\textsc{maq} were a lexically  biased question, that is, if a +\textsc{maq} obligatorily denoted the speaker's bias towards the positive answer, then \xref{xiayu1} and \xref{xiayu3} would be unacceptable since the continuations contradict the semantic content of the +\textsc{maq} .

\begin{exe}
\ex\label{xiayu} \gll Xia yu le ma? ...\\
fall rain \textsc{perf} ma\\
\glt `Did  it rain? ...'
\begin{xlist}
\ex\label{xiayu1} ... \gll Wo bu juede.\\
I \textsc{neg} think \\
\glt `... I don't think so.'
\ex\label{xiayu2} ... \gll Wo cai xia le.\\
I guess fall \textsc{perf}\\
\glt `... I guess it rained.'
\ex\label{xiayu3} ... \gll Wo wanquan bu qingchu.\\
I totally \textsc{neg} clear\\
\glt `... I totally have no idea.' \jambox*{(+\textsc{maq})}
\end{xlist}
\end{exe}

Furthermore,  the speaker does not need to be neutral about the answers when she utters an \textsc{anaq}.
  The speaker  in \xref{neutral} is probably neutral  regarding the answers, while  she can hold a private bias toward either answer in some other contexts.  For instance, A in \xref{wenti} can felicitously utter an \textsc{anaq} even though she is privately biased towards the positive alternative \textit{You have questions}:


\begin{exe}
\ex\label{wenti}  A believes that his audience usually have questions to raise after his speech:
\begin{xlist}
\exi{A:} \gll Nimen you mei you wenti?$\downarrow$\\
you have not have questions\\
\glt `Do you have or not have questions?'\jambox*{(\textsc{anaq})}
\end{xlist}
\end{exe}


Similarly in \xref{portrait}, A is only privately biased towards the proposition \textit{He looks like you}, that is, the other discourse participant does not know that A is biased towards p.\footnote{We owe example \xref{portrait} to an anonymous reviewer.}

\begin{exe}
\ex\label{portrait} A and B are in the museum. A finds a portrait that looks like B.
\begin{xlist}
\exi{A:} \gll (Kan!) Ta xiang bu xiang ni?\\
 look he resemble not resemble you\\
\glt `(Look!) Does he look like you or not like you?'
\end{xlist}
\end{exe}



Finally, the notion of the speaker bias seems  too strong to characterize the bias that arises from $-$\textsc{maq}s.  That is, the speaker does not need to have a strong belief that $\neg p$ in uttering a  $-$\textsc{maq}. In \xref{xiayune3}, for instance, speaker A only considers a (slight) possibility of $\neg p$ or just reports that someone else other than the speaker mentioned $\neg p$, yet the context licences the use of  the $-$\textsc{maq}.


\ea\label{xiayune3}
\begin{xlist}
\exi{A1:} \gll Zuowan keneng mei xia yu. \\
last-night maybe \textsc{neg} fall rain \\
\glt `Maybe it did not rain last night.'
\exi{A2:} \gll Wo bu juede zuowan xia yu le.\\
I \textsc{neg} think last-night fall rain \textsc{perf} \\
\glt `I don't think that it rained last night.'
\exi{A3:} \gll John shuo zuowan mei xia yu. \\
John say last-night \textsc{neg} fall rain \\
\glt `John said that it did not rain last night.'
\exi{B:} \gll Mei xia yu ma? \\
\textsc{neg} fall rain ma\\
\glt `Did it not rain?' \jambox*{($-$\textsc{maq})}
\end{xlist}
\z

In summary, the  speaker bias is not in the lexical specification of either \textsc{maq}s or \textsc{anaq}s.  As we argue below,  the notion of `contextual bias' is more appropriate to characterize  the  semantics and pragmatics of \textsc{maq}s and \textsc{anaq}s.


\section{Proposals}

To derive the distribution of the polar questions sketched in \sectref{sec:data}, we make the following proposals.

\ea
\ea The bias meaning involved in Mandarin polar questions are best characterized by the notion of contextual bias (`evidential bias' in \citealt{Sudo})
	\ex As for \textsc{maq}s, only the negative \textsc{maq} lexically encodes the bias meaning.
	\ex The contextual neutrality of \textsc{anaq} is derived by the exhaustivity operator denoted by the final low tone $\downarrow$/L\%.
\z
\z




 When the context is biased toward $\neg p$, a  $-$\textsc{maq} is the most optimal.  Accordingly, a +\textsc{maq} is used  elsewhere, i.e., in neutral and positively biased contexts as we have seen in \tabref{matable1}, repeated here as \tabref{maqtab2}.  In \sectref{subsec:pnmaq}, we show how the elsewhere condition  explains the complementary distribution of positive/negative \textsc{maq}s.


\begin{table}
	\begin{tabularx}{\textwidth}{lCCc}
	\lsptoprule
		&  Neutral  & biased towards  $p$ & biased towards $\neg p$\\
	\midrule
		positive +\textsc{maq}s & \cmark & \cmark & \#\\
		negative -\textsc{maq}s & \# & \# & \cmark\\
\lspbottomrule
	\end{tabularx}
\caption{Distribution of \textsc{maq}s}
\label{maqtab2}
\end{table}
	
	
As for \textsc{anaq}s, we argue that the exhaustive interpretation that arises from the final low tone $\downarrow$/L\% is the source of the neutrality requirement.  In a nutshell,  $\downarrow$ expresses that the Hamblin alternatives presented by the A-not-A construction, i.e., $p$ and  $\neg p$, are the only live options.  Thus, if the context is biased towards one of them, the context does not match the semantics of  $\downarrow$.
	
In the next sections, we formalize the notion of contexual bias/neutrality and show how the distribution pattern of Mandarin polar questions can be derived.


\section{Formalizing contextual bias}

In formalizing the contextual bias,  we start with the following working definition:
A context ${c}$ is biased toward $p$ when someone's bias  towards $p$ is public:

\ea Contextual bias  (informal version)\\
A context ${c}$ is biased toward $p$ iff
	\ea\label{entertain} it is a common belief that {some individual $x$ entertains the possibility of}  $p$, and
\ex\label{commonb}  there is no individual who entertains the possibility of $\neg p$. 
	\z
	\z

 
 To characterize some individual's epistemic state, we use the notion of subjective probability distribution \citep{Jeffrey2004, Potts, MO2007, davis2007pragmatic}.  
In implementing the ``common belief'' part, we employ \quotecite{FB} Table model.



\subsection{Subjective probability distribution}    

The current paper follows  the formulation given by \citet{davis2007pragmatic} and  models a proposition (i.e., a set of possible worlds) as  a  probability distribution:



 \ea A probability distribution for a countably finite set $W$ is a function $P^{W}$ from subsets
of $W$ into real numbers in the interval [0,1] obeying the conditions:%\footnote{Since there is no uniform distribution over a countably infinite set, $W$ is assumed to be a finite set for simplicity.}
\ea
$P^{W}$($W$) = 1
\ex $P^{W}$(\{$w$\}) $\geqslant$ 0 for all $w \in W$
\ex If $p$ and $q$ are disjoint
subsets of $W$, then $P^{W}$($p$ $\cup$ $q$) = $P^{W}$($p$)+$P^{W}$($q$).\footnote{We henceforth suppress the superscript $W$.}\\ \jambox*{(\citealt{davis2007pragmatic}: 77)}
\z
\z




 The epistemic state of an individual $a$ is denoted by a proposition $\mathrm{Dox}_a$, which is a finite set of possible worlds that are doxastically accessible to $a$.  Now, the conditionalization of a uniform distribution is given in \xref{cprod}.


\ea\label{cprod} Let $P(-|p)$ be the function that maps any proposition $q$ to
    \begin{displaymath}
P(q|p) = \frac{P(q \cap p)}{P(p)}
\end{displaymath}
  where $P$ is a probability distribution. That is, $P(-|p)$ maps propositions to
their conditional probabilities (for $P$) given $p$. $P(q|p)$ is  undefined if $P(p) = 0$.\jambox*{(\citealt{davis2007pragmatic}: 77)}
\z


Based on  this uniform distribution,  a function $\mathrm{Cred}_{a}$ ($\mathrm{Cred}$ for `credence')  models the epistemic state of an individual $a$ as in \xref{credence}.
The function $\mathrm{Cred}_a$  returns $a$'s degree of belief in $p$:

\ea\label{credence} The subjective probability distribution for an individual $a$:
\begin{displaymath}
\mathrm{Cred}_a = P(-|\mathrm{Dox}_a)
\end{displaymath}
in which $P$ is a uniform distribution over $W$, i.e., $P(\{w\}) = \frac{1}{|W|}$ for all $w  \in
W$.\jambox*{(Modified from  \citealt{davis2007pragmatic}: 77)} 
\z


Thus, an individual $a$'s degree of belief in a proposition $p$ is calculated as follows: 


\ea\label{credencep} 
\begin{displaymath}
\mathrm{Cred}_a({p}) = P({p}|\mathrm{Dox}_a) = \frac{P({p} \cap \mathrm{Dox}_{{a}})}{P(\mathrm{Dox}_{{a}})}
\end{displaymath}
\z


  Now let us calculate   different belief states using \xref{credencep}. If  $a$ is committed to the proposition $p$, $p$ is true in all the  worlds in $\mathrm{Dox}_{a}$, i.e., $ \mathrm{Dox}_{{a}}\subseteq p$.  Since $p \cap  \mathrm{Dox}_{{a}} = \mathrm{Dox}_{a}$, $\mathrm{Cred}_{{a}}(p)$ returns 1:

  \ea  \begin{displaymath}
\mathrm{Cred}_{a}(p) = P(p|\mathrm{Dox}_{a}) = \frac{P(p \cap \mathrm{Dox}_{a})}{P(\mathrm{Dox}_{a})} = \frac{P(\mathrm{Dox}_{a})}{P(\mathrm{Dox}_{a})} = 1
\end{displaymath}
\z

 If $a$ is committed to $\neg p$, $p$ is true in no  worlds in $\mathrm{Dox}_{a}$. Thus, $\mathrm{Cred}_{a}(p)=0$:

  \ea  \begin{displaymath}
\mathrm{Cred}_{a}(p) = P(p|\mathrm{Dox}_{a}) = \frac{P(p \cap \mathrm{Dox}_{a})}{P(\mathrm{Dox}_{a})} = \frac{\frac{0}{|W|}}{P(\mathrm{Dox}_{a})} = 0
\end{displaymath}
\z


Finally, agent  $a$ entertains the possibility of $p$ when $\mathrm{Cred}_a(p)$ is greater than 0:

\ea\label{33} $a$ entertains the possibility of $p$   iff
\begin{displaymath}
\mathrm{Cred}_{a}(p) > 0
\end{displaymath}
where $a\in  A$ and $A$ is a set of epistemic agents.
\z

Let us see how \xref{33} works with linguistic examples.  In \xref{4.56wr} the speaker is committed to the proposition $p$ \textit{It rained}. That is,  $\mathrm{Cred}_{\mathrm{spkr}}(p)=1>0$, so the speaker entertains the possibility of $p$.


\ea\label{4.56wr}
 \gll Zuowan xia yu le.\\
last.night fall rain \textsc{perf}\\
\glt `It rained last night.'
\z


Similarly, in \xref{Johnwr1},  John believes $p$ to be true, so $\mathrm{Cred}_\mathrm{john}(p)\geqslant 0.98>0$, thus John entertains the possibility of $p$.

\ea\label{Johnwr1}
\gll John shuo/juede zuowan xia yu le.\\
John said/belives last.night fall rain \textsc{perf}\\
\glt `John said/believes that it rained last night.'
\z


In \xref{4.55w}, the fact that B wears a wet raincoat is regarded as contextually compelling evidence \citep{BG} for the proposition $p$ \textit{It rained} that raises A's degree of belief in $p$ to a value greater than 0.5 ($1 > \mathrm{Cred}_{\mathrm{A}}(p) > 0.5$, see \citealt{MO2007}).   Since $ \mathrm{Cred}_{\mathrm{A}}(p) > 0$, agent $A$ entertains the possibility of $p$.


\ea \label{4.55w} B enters A's windowless room wearing a  wet raincoat.
\z


This section formalized the ``entertaining'' part of the definition of contextual bias  using the subjective probability distribution.  Agent $a$ entertains the possibility of $p$ when $ \mathrm{Cred}_{a}({p})$ is greater than $0$ as defined in \xref{33}.

We next turn to the rest of the definition, that is, how someone's epistemic state becomes a common belief of all conversation participants.

\subsection{The Table model}\label{subsec:tab}
We follow \quotecite{FB} idea that when an issue that contains a proposition is pushed onto the conversation ``Table'', the proposition becomes a common belief.  Thus, in our case,  the context is biased towards $p$ when an issue that contains the proposition that some individual entertains the possibility of $p$ is pushed onto the Table, and no issue that contains $\neg p$ is  pushed onto the Table.



 The Table is one way to represent Questions Under Discussion \citep{Ro} and defined as in \xref{defT}. Let $I$ be an issue, a set of propositions of type  $\langle\langle s, t\rangle, t\rangle$.  A Table $T$ is a stack or an ordered pair of issues.  






 \ea \label{defT} The Table $T$:\\
 Let $I$ be an issue, a set of propositions.
 \ea $\langle\, \rangle$ is a Table.
\ex If $I$ is an issue and $T$ is a Table, then $\langle I, T\rangle$ is a Table.
\ex Nothing else is a Table.
\ex If $T$ is a Table, then $|T|$ is the length of the table and $T[{n}]$ is the ${n}$th element in the Table ($1\leq n \leq |T|$; counting from 1 at the top).

\z
\z

If the Table is  not empty, there is some issue to be solved.  The topmost issue on the Table is the most-pressing issue that needs to be resolved.   The ultimate goal of the conversation is to resolve all issues and empty the Table.

  Stack operations such as $\mathrm{push}$ and $\mathrm{pop}$ are also introduced as  operations for the Table in \citet{FB}. Performing $\mathrm{push}(I, T)$ outputs a new stack  by adding  $I$ to
the top of the stack $T$:



\ea\label{oplus} For any issue $I$ and Table $T$: \\$\mathrm{push}(I, T) = \langle I, T\rangle$
\z

The `$\mathrm{pop}(I, T)$' operation  removes the topmost issue $I$ from $T$:

\ea\label{defpop} For any issue $I$ and Table $T$: \\$\mathrm{pop}(I, T) =  T$ if $T\neq \langle\,\rangle$; $I$ otherwise.
\z

% If one proposition $p$ in the set  $I$ enters the \textsc{cg}, the issue $I$ is resolved and hence removed from the Table.

Each Table is relativized to a context $c$, which has a basic semantic type $\textsc{c}$   (see also \citealt{Davis2}).\footnote{In \quotecite{FB} framework,  a context state is understood as  a tuple of elements such as the Common Ground, the Table, etc (see also \citealt{RF2015}).  Speech act operators such as \textsc{assertion} and \textsc{question} take sentences as
arguments and yield functions from  input context states to  output context states.}  Thus,  $T$ is now a function from contexts to Tables:

\ea The Table in context:\\
Let $c$ be a context, $T(c)$ is a Table at context $c$.
\z

Similarly, the Stalnakerian \citeyearpar{Stalnaker} Common Ground is obtained by a function $\textsc{cg}$ that takes a context $c$ and returns a set of propositions:

\ea The Common Ground:\\
Let $c$ be a context,
$\textsc{cg}(c)$ is a set of propositions that are shared by all the discourse participants at context $c$.
\z

 

Speech acts are defined as functions from input contexts to  output contexts.   The \textsc{assert}  operator is of type $\langle\langle s, t\rangle, \langle\textsc{c}, \textsc{c}\rangle\rangle$ and it takes a proposition  $p$ and yields a context change potential of type $\langle\textsc{c}, \textsc{c}\rangle$:

\ea \label{bf3} \textsc{ccp} of \textsc{assert}  (first version):\\ $\textsc{assert}(p)({c}) = {c}'$ such that
\ea\label{bf31} $\textsc{cg}({c}') = \textsc{cg}({c})  \cup (\mathrm{Cred}_\mathrm{spkr}(p) \geqslant 0.98)$
\ex\label{bf32}
 ${T}({c}') = \mathrm{push}(\{p\}, {T}(c))$
 \z
\z

  As can be seen in \xref{bf3},  an assertion of $p$ updates the context in two ways. First,  it adds to the $\textsc{cg}(c)$ a proposition that   the speaker has a very high degree of belief in $p$  ($Cred_{\mathrm{spkr}}(p)\geqslant 0.98$). Second, it pushes $\{p\}$  onto the top of the Table.\footnote{The  second update on the Table will be dispensable as the first part automatically pushes $p$ to the Table, so \xref{bf32} will be removed later.}   This second update is one of the core features of \quotecite{FB} Table model. In \xref{respondt},  A's assertion of $p$ \textit{It rained} is  directly assented or dissented with by B.  In other words, not only a question but but also an assertion can raise an issue. Thus, as soon as $p$  is asserted, it is  considered as an at-issue proposition on the Table that affects the future direction of the discourse \citep{FB, Tonhauser, Northrup}.

 \ea \label{respondt}
 \begin{xlist}
 \exi{A:} \gll Zuowan xia yu le.\\
last-night fall rain \textsc{perf} \\
\glt `It rained last night.'
\exi{B:} \gll Shide, xia yu le. / Bu-shide, mei-you xia yu.\\
yes fall rain \textsc{perf} {} \textsc{neg}-yes \textsc{neg}-have fall rain \\
\glt `Yes, it rained'/`No, it did not rain.'
\end{xlist}
 \z


Besides   conversational moves such as an assertion
or a question that are extensively discussed in \citet{FB}, we propose that  contextual compelling
evidence is another conversational move that affects the context. For instance in \xref{4.55wr}, the fact that B wears a raincoat counts as evidence for the proposition $p$ \textit{It rained} which in turn increases A's degree of belief that $p$ to some degree above 0.5.

\ea\label{4.55wr} B enters A's windowless room wearing a  wet raincoat.
\z

Thus,  contextual compelling evidence yields a   context change potential. We define  $\textsc{cce}$ based on \quotecite{MO2007} semantics of Japanese evidentials as in \xref{Efr}. The operator $\textsc{cce}$ presupposes that some evidence $q$ has led $a$ to raise her subjective probability of $p$ above 0.5.  For example, suppose that $a$ holds a background knowledge $q\to p$ `\textit{If someone wears a wet raincoat, it is raining}'.  When $a$ learns that $q$ is true, by modus ponens, the probability of $p$ ($P(p|\mathrm{Dox}_a\cap q)$) becomes higher than before leaning $q$ and 0.5.    If the presupposition is satisfied,  $\textsc{cce}$ combines  with the proposition  $p$ and returns a \textsc{ccp}, which changes the context ${c}$ by adding the proposition that `\textit{A's degree of belief in $p$ is larger than 0.5}' into the $\textsc{cg}({c})$ and pushing the issue of $p$ onto the Table ${T}$(${c}$).\footnote{\xref{Ef32} will  be removed later as $a$'s high credence on $p$ being part of the common ground is enough for $p$ to be an issue on the Table.}



\ea\label{Efr} \textsc{ccp} of $\textsc{cce}$ (contextual compelling evidence)  (first version):\\
Let $p, q$ be  propositions  and $a$ be a discourse participant,
\ea $\textsc{cce}_a(p)({c})$ is defined iff $\exists q.  P(p|\mathrm{Dox}_a\cap q) > P(p|Dox_a) \land P(p|\mathrm{Dox}_a\cap q)> 0.5$
\ex If defined,
$\textsc{cce}_a(p)({c}) = {c}'$ such that
\ea\label{Ef31} $\textsc{cg}({c}') = \textsc{cg}({c})  \cup (\mathrm{Cred}_a(p) > 0.5)$
\ex\label{Ef32}
 ${T}({c}') = \mathrm{push}(\{p\}, {T}({c}))$
\z
\z
\z

\largerpage
It follows that  $p$ is at-issue and  up for debate, just like the issues pushed onto the Table by prototypical conversational moves such as assertion and question. Therefore, discourse participants can  respond to a piece of contextual compelling evidence for the proposition $p$ \textit{It rained} by showing their agreement or disagreement with $p$, as illustrated in \xref{4.55wrrr}.   Also, A's use of the anaphoric expression \textit{zheyang} `this' in \xref{4.55wrrr} referring to  $p$ (`\textit{I expected it rained}'/`\textit{I don't believe it rained}') demonstrates that the existence of the contextual compelling evidence for $p$ enables $p$  to be the antecedent of the anaphor. This  is possible because the contextual compelling evidence for $p$  raises an issue $\{p\}$  that can be discussed in the subsequent discourse (see \citealt{Snider2017} for the discussion of  at-issueness and anaphoric salience).

\ea\label{4.55wrrr} B enters A's windowless room wearing a  wet raincoat.
\begin{xlist}
\exi{A:} \gll Wo jiu zhidao (hui zheyang). \\
I \textsc{part} know can this-like\\
\glt `This is what I expected.'
\exi{A$^{\prime}$:} \gll Bu keneng, wo bu xiangxin (hui zheyang).\\
\textsc{neg} possible I \textsc{neg} believe would this \\
\glt `No way, I don't believe this (would happen).'
\end{xlist}
\z



As can be seen from \xref{respondt} and \xref{4.55wrrr},   as long as some individual  publicly entertains the possibility of   $p$, $p$ becomes an issue that is  on the Table for discussion. To implement this intuition, we  propose that as long as
 some individual's  consideration of the possibility of $p$ is made public in a context ${c}$,  the issue $\{p\}$ is pushed onto the Table at  ${c}$:

      \ea \label{tablecg} Pushing an issue onto the Table:\\
      If $\textsc{cg}({c}') =\textsc{cg}({c}) \cup (\exists x.x\in A({c})\&\mathrm{Cred}_{x}(p) > 0),$\\ Then
 ${T}({c}') = \mathrm{push}(\{p\}, {T}({c}))$.\\ where ${c}'$ and ${c}$ are  the output context and  input context respectively and $A({c})$ is the set of epistemic agents at ${c}$.
\z

 Now that \xref{tablecg}  allows an issue $\{p\}$ to be on the Table as long as   some individual   considers $p$ possible,  the definitions of  \textsc{assert} and $\textsc{cce}$ are simplified as below:


\ea\label{bf3r} \textsc{ccp} of \textsc{assert}  (final version):\\$\textsc{assert}(p)(c) = \textsc{c}'$ such that
$\textsc{cg}(c') = \textsc{cg}(c)  \cup (\mathrm{Cred}_\mathrm{spkr}(p) \geqslant 0.98)$
\z


\ea \label{Efrr} \textsc{ccp} of $\textsc{cce}$  (final version):\\ 
Let $p, q$ be  propositions  and $a$ be a discourse participant,
\ea $\textsc{cce}_a(p)({c})$ is defined iff $\exists q. P(p|\mathrm{Dox}_a\cap q) > P(p|Dox_a)\land P(p|\mathrm{Dox}_a\cap q)>0.5$
\ex If defined,
$\textsc{cce}_{a}(p)(c) = \textsc{c}'$ such that $\textsc{cg}(c') = \textsc{cg}(c) \cup (\mathrm{Cred}_{a}(p) > 0.5)$
\z
\z

Finally, we can formalize the contextual bias and neutrality.  First, A context ${c}$ is biased towards a proposition $p$ if  the issue $\{p\}$, but not $\{\neg p\}$, is on the Table in~${c}$:

\ea\label{cbtable} Contextual bias (final version)\\
 A context ${c}$ is biased towards a proposition $p$ iff \\
 $\{p\}\subseteq\bigcup_{x=1}^{n}{T}({c})[x]$ and $\{\neg p\}\not\subseteq\bigcup_{x=1}^{n}{T}({c})[x]$,\\
 where $n=|T({c})|$.
\z

We also define context neutrality as follows: The context is neutral with respect to $p$ if no issue is on the Table or if both $p$ and $\neg p$ are on the Table.  More precisely, the context is neutral when the union of all the issues at each stack member of the Table amount to  an empty set  or contains the issue $\{p, \neg p\}$:

\ea The context {c} is neutral with respect to $p$ iff $\bigcup_{x=1}^{n}{T}({c})[x]  = \emptyset$ or $\{p, \neg p\}\subseteq\bigcup_{x=1}^{n}{T}({c})[x]$,\\
 where $n=|T({c})|$.
\z

To sum up, we formalize the notion of contextual bias using the subjective probability and the Table model.  A context is biased towards $p$ when the proposition that someone entertains the possibility of $p$ becomes an  issue of the Table.  


\section{Semantics of Mandarin polar questions}
Before looking at how our notion of contextual bias derives the pattern summarized in \sectref{sec:data}, we briefly review the semantics of  \textsc{maq} and \textsc{anaq} given by \citet{YuanHaraGlowinAsia2019}.  Furthermore, this paper adds felicity conditions on the semantics of $-$\textsc{maq}.


\subsection{\textit{Ma} Questions}

\citet{YuanHaraGlowinAsia2019} claim that a \textsc{maq} like \xref{441} has the syntactic structure given in \xref{matree}.


\ea\label{441} \gll Li he jiu ma?\\
Li drink alcohol $\textsc{q}_{1}$\\
\glt `Does Li drink alcohol?'
\z


\ea \label{matree}
\begin{forest}
 [ForceP [TP [Li] [VP [he jiu , roof ] ] ] [Force [\textit{ma}/$\textsc{q}_{1}$ ] ] ] 
 \end{forest}
\z


The semantics of the particle \emph{ma}/$\textsc{q}_{1}$ is defined as in \xref{q}.  The particle takes its sister proposition $p$ as an argument, creates a Hamblin alternative $\{p, \neg p\}$ as an issue and pushes the issue onto the Table:

\ea\label{q} 
\textsc{ccp} of  $\textsc{q}_{1}$:\\
$\textsc{q}_{1}(p)({c}) = {c}'$ such that  ${T}({c}') = \mathrm{push}(\{p, \neg p\}, {T}({c}))$
\z

Now, we claim that $-$\textsc{maq}s have a felicity condition \citep{Searle} in addition to the usual felicity condition of questions that  +\textsc{maq}s also have.\footnote{See \citet{Trinh2014} for an alternative analysis of the felicity conditions for positive and negative polar questions. \citet{Trinh2014} makes two generalizations about the felicitous use of polar questions: 1. A polar question is felicitous only if its prejacent does not contradict the answer implied
by contextual evidence; 2. In contexts where there is neither evidence for p nor evidence for $\neg p$, the question denoting $\{p, \neg p\}$ is felicitous only if it is an inverted positive question.  \citet{Trinh2014}  explains the first generalization  by adopting the Principle of Maximize Presupposition and explains the second one by adopting the Maxim of Manner: in a neutral context where there is neither evidence for p nor evidence for $\neg p$, the speaker will choose a positive polar question instead of a negative one because the former is simpler  in syntactic form.}  In a nutshell, a $\neg p$-\emph{ma} is felicitous only when the context is biased towards $\neg p$, i.e.,  $\{\neg p\}\subseteq\bigcup_{x=1}^n{T}({c})[x]$ and $\{ p\}\not\subseteq\bigcup_{x=1}^n{T}({c})[x]$:




\ea\label{qn2} Felicity condition of $-\textsc{maq}$:\\
The use of a negative $\textsc{maq}$, i.e., a $\textsc{maq}$ containing NegP as the maximal
I-projection which denotes $\neg p$, is felicitous in a context $\textsc{c}$ only if $\textsc{c}$ is
biased towards $\neg p$.
\z


\subsection{A-not-A questions}\label{subsec:semanaq}
Turning to \textsc{anaq}s, \citet{YuanHaraGlowinAsia2019} propose that an \textsc{anaq} like \xref{Type} has the structure in \xref{Tree}, adopting \quotecite{HuangA1991} analysis.  The feature \textsc{r} represents  the reduplication of the predicate with the negative marker \emph{bu}.  The optional particle \textit{ne} is the phonological realization of another question operator $\textsc{q}_{2}$.



\ea\label{Type} \gll Li he bu he jiu (ne)?$\downarrow$\\
Li drink not drink alcohol $\textsc{q}_{2}$\\
\glt `Does Li drink or not drink alcohol?'
\z

\ea\label{Tree} 
\begin{forest}
[ForceP [TP [$\mathrm{NP}_{1}$ [Li , roof] ] [T$'$ [T [\textsc{r}] ]  [VP [V [ he ] ] [$\mathrm{NP}_{2}$ [jiu , roof ] ] ] ] ] [Force [$\textsc{q}_{2}$/ne ] ] ] 
\end{forest}
\z


The reduplication feature  \textsc{r} is responsible for creating  a Hamblin set:

\ea\label{qde} Semantics of reduplication \textsc{r}\\
$\llbracket\textsc{r}\rrbracket = \lambda P. \lambda x. \{P(x), \neg P(x)\}$
\z

The particle $\textsc{q}_{2}$/\emph{ne} pushes the Hamblin set created by  \textsc{r} to the Table:

\ea\label{Q2} \textsc{ccp} of the operator $\textsc{q}_{2}$:\\
$\textsc{q}_{2}(Q)(c) = \textsc{c}'$ such that  ${T}(\textsc{c}') = \mathrm{push}(Q, {T}(c))$
\z


Finally,  an \textsc{anaq} has to be uttered with a final L\% boundary tone ($\downarrow$).  Following \quotecite{BR} analysis of English final falling tone of alternative questions, \citet{YuanHaraGlowinAsia2019} argue that the Mandarin final $\downarrow$/L\%  is a closure operator which indicates that there is no issue on the Table or the issue presented by the  \textsc{anaq}  is the only issue on the Table.  The current paper proposes the following semantics for $\downarrow$:\footnote{An \textsc{anaq} followed by the particle \textit{ne} can be uttered with final H\%, as pointed out by an anonymous reviewer.  We speculate that when  an \textsc{anaq} containing \textit{ne} is uttered with H\% tone, the L\% tone is overridden by H\%.  Thus, the source of neutrality is still L\%.    Our intuition is that when an \textsc{anaq} containing \textit{ne} is uttered with H\% tone, the speaker is more anxious to know the answer, compared with \textsc{anaq}s without \textit{ne}. This is why \citet{Shao} argues that the semantics of \textit{ne} reinforces the interrogative force. We believe that this reinforcing meaning is not due to  \textit{ne}, but due to the H\% boundary  tone. Following \citet{Bartels} and \citet{HD}, the H\% tone indicates that the utterance is directed at the addressee and  the speaker expects   the addressee to resolve the issue. Thus, when uttering \textsc{anaq}s containing \textit{ne} with H\% tone, the speaker sounds more anxious  in seeking an answer. Another complexity is that whether the particle \emph{ne} carries the final H\% or  L\% depends  on the lexical tone of the previous syllable.  Thus, in \xref{chaung} when the previous syllable carries a high level lexical tone (55), the particle \emph{ne} cannot carry a H\%. 

\ea\label{chaung}\gll Kai55 bu kai55 chuang55 ne?\\
     open  not open  window   ne\\
    \glt `Shall we open the window or not?'
  \z  
    
}

\ea\label{dl} Semantics of $\downarrow$\\
$\llbracket\downarrow\rrbracket = \lambda \varphi. \lambda {c}. \bigcup_{x=1}^n{T}({c})[x]  = \varphi$ or $\emptyset$,\\
where $n=|T({c})|$
\z

Thus, when an \textsc{anaq} (i.e., $p$\emph{-or-not-}$p$) is uttered with $\downarrow$, it expresses that the Table has no issue or that only issue on the Table is $\{p, \neg p\}$.


\section{Deriving the distribution}

 Let us illustrate how our notion of contextual bias \xref{cbtable}, repeated here as \xref{cbtable2}, together with the notion `pushing an issue on the Table' \xref{tablecg}, repeated here as \xref{tablecg2}, correctly predicts the felicity of \textsc{maq}s and \textsc{anaq}s in different contexts.
 
 \ea\label{cbtable2} Contextual bias (final version)\\
 A context \textsc{c} is biased towards a proposition $p$ iff\\
 $\{p\}\subseteq\bigcup_{x=1}^n{T}({c})[x]$ and $\{\neg p\}\not\subseteq\bigcup_{x=1}^n{T}({c})[x]$,\\
 where $n=|T({c})|$.
\z
 
  \ea \label{tablecg2} Pushing an issue onto the Table:\\
      If $\textsc{cg}({c}') =\textsc{cg}({c}) \cup (\exists x.x\in A({c})\ \&\ \mathrm{Cred}_{x}(p) > 0),$\\ Then
 ${T}({c}') = \mathrm{push}(\{p\}, {T}({c}))$.\\ where ${c}'$ and ${c}$ are  the output context and  input context respectively and $A({c})$ is the set of epistemic agents at ${c}$.
\z
 
\subsection{Positive/Negative \textit{ma} questions}\label{subsec:pnmaq}
As summarized in \sectref{sec:data}, +\textsc{maq}s and $-$\textsc{maq}s are in complementary distribution.
  Recall that a $\neg p$-\emph{ma} has a  specific felicity condition \xref{qn2} that dictates that the Table must contain the issue $\{\neg p\}$.  On the other hand, positive \textsc{maq}s do not have such a contextual requirement.   \textsc{maq}s.  That is, $p$-\emph{ma} is unacceptable when the context is biased towards $\neg p$ while it is acceptable when the context is neutral or biased towards $p$.  We show in this section that the distribution is straightforwardly explained in terms of pragmatic competition.
 
 
 \subsubsection{Neutral context}
 In \xref{dii2}, there is no issue on the Table, thus $-$\textsc{maq} is ruled out, while $+$\textsc{maq}, which does not have such an extra condition, is okay:
 
\ea\label{dii2}The first question in a questionnaire investigating the relationship between  weather and people's mental states is:
\begin{xlist}
\exi{Q:}[]{\gll Ni de chengshi zuotian  xia yu le ma?\\
you \textsc{gen} city yesterday fall rain \textsc{perf} ma \\
\glt `Did it rain yesterday in your city?'\jambox*{($+$\textsc{maq})}}
\exi{Q$'$:}[\#]{\gll Ni de chengshi zuotian mei xia yu  ma?\\
you \textsc{gen} city yesterday \textsc{neg} fall rain ma \\
\glt `Did it not rain  yesterday in your city?'\jambox*{($-$\textsc{maq})}}
\end{xlist}
\z

\subsubsection{positively biased context}\label{subsubsec:pos}

As shown  in \xref{4.56wr}, repeated here as \xref{4.56wrr},  A  has asserted the proposition $p$ \textit{It rained}. That is, $\textsc{cg}({c}') = \textsc{cg}({c}) \cup (\mathrm{Cred}_{\mathrm{A}}(p) \geqslant 0.98)$, hence the issue $\{p\}$ is pushed onto the Table and the context is biased towards $p$. A default +\textsc{maq} is felicitous in such a positively biased context.  In contrast,  $\neg p$-\emph{ma} is infelicitous since unlike  +\textsc{maq}s, $-$\textsc{maq}s have a contextual requirement that the context needs to be negatively biased.

\ea\label{4.56wrr}
\begin{xlist}
\exi{A:}[]{Zuowan xia yu le.
\glt `It rained last night.'}
\exi{B:}[]{\gll Xia yu le ma?\\
fall rain \textsc{perf} ma\\
\glt `Did it rain?' \jambox*{($+$\textsc{maq})}}
\exi{B':}[\#]{\gll Mei xia yu ma?\\
\textsc{neg} fall rain ma\\
\glt `Did it not rain?' \jambox*{($-$\textsc{maq})}}
\end{xlist}
\z


The assertion of bare $p$ is not the only way to mark the context as biased towards $p$, but a modalized or embedded $p$ as in A1--A3 of \xref{weakb}  is enough to make the context $p$-biased.  Recall from the definition of `pushing an issue onto the Table' \xref{tablecg} that $p$ is pushed onto the Table as long as there is some individual $x$, who is not necessarily a conversation participant, that entertains the possibility of $p$ ($\exists x.x\in A({c})\ \&\ \mathrm{Cred}_{x}(p) > 0$).  Thus, since all the A-utterances in  \xref{weakb}   make the context biased towards $p$, only the $+$\textsc{maq} is felicitous:

\ea\label{weakb}
\begin{xlist}
\exi{A1:}[]{\gll Zuowan keneng  xia yu le. \\
yesterday possible fall rain \textsc{perf}\\
\glt `Maybe it  rained last night.'\label{keneng}}
\exi{A2:}[]{\gll Wo  juede zuowan xia yu le.\\
I think last.night fall rain \textsc{perf}\\
\glt `I  think that it rained last night.'}
\exi{A3:}[]{\gll John shuo zuowan  xia yu le. \\
John said last.night  fall rain \textsc{perf}\\
\glt `John said that it rained last night.'\label{john}}
\exi{B:}[]{\gll Xia yu le ma?\\
fall rain \textsc{perf} ma \\
\glt `Did it rain?'\jambox*{($+$\textsc{maq})}}
\exi{B':}[\#]{\gll Mei xia yu  ma? \\
\textsc{neg} fall rain ma\\
\glt `Did it not rain?' \jambox*{($-$\textsc{maq})}}
\end{xlist}
\z

Similarly, in \xref{rc} the contextually compelling evidence pushes $p$ onto the Table (see \xref{Efrr} and \xref{tablecg2}), thus the context is biased towards $p$ and  $p$-\emph{ma} is okay while $\neg p$-\emph{ma} is unacceptable:

\ea\label{rc} B enters A's windowless  room wearing a wet raincoat.
\begin{xlist}
\exi{A:}[]{\gll Xia yu le ma?\\
fall rain \textsc{perf} ma \\
\glt `Did it rain?'\jambox*{($+$\textsc{maq})}}
\exi{A$'$:}[\#]{\gll Mei xia yu  ma? \\
\textsc{neg} fall rain ma\\
\glt `Did it not rain?' \jambox*{($-$\textsc{maq})}}
\end{xlist}
\z

In short, a $-$\textsc{maq}  cannot be used in neutral nor positively biased contexts as it has a felicity condition that requires negatively biased contexts. A +\textsc{maq} does not have such a requirement, thus it is a default polar question that can be used in both neutral and positively biased contexts. 


\subsubsection{negatively biased contexts}\label{subsubsec:neg}
Now, let us look at the contexts where $-$\textsc{maq}s are used.   As long as the context suggests that someone entertains the possibility of $\neg p$ as in the following, $\neg p$ is pushed onto the Table and  the context is biased towards $p$.  Since this is the context that the felicity condition of $\neg p$-\emph{ma} \xref{qn2} requires, $-$\textsc{maq} wins over +\textsc{maq} as a result of pragmatic competition:

\ea\label{xiayune2}
\begin{xlist}
\exi{A1:}[]{\gll Zuowan (keneng) mei xia yu. \\
last-night possible \textsc{neg} fall rain \\
\glt `(Maybe) it did not rain last night.'}
\exi{A2:}[]{\gll Wo bu juede zuowan xia yu le.\\
I \textsc{neg} think last-night fall rain \textsc{perf} \\
\glt `I don't think that it rained last night.'}
\exi{A3:}[]{\gll John shuo zuowan mei xia yu. \\
John say last-night \textsc{neg} fall rain \\
\glt `John said that it did not rain last night.'}
\exi{B:}[]{\gll Mei xia yu  ma? \\
\textsc{neg} fall rain ma \\
\glt `Did it not rain?' \jambox*{($-$\textsc{maq})}}
\exi{B$'$:}[\#]{\gll Xia yu le ma?\\
fall rain \textsc{perf} ma \\
\glt `Did it rain?'\jambox*{($+$\textsc{maq})}}
\end{xlist}
\z


Similarly, when the contextually compelling evidence supports $\neg p$, $\neg p$ is pushed onto the Table, thus the context is biased towards $\neg p$.  Thus,  $\neg p$-\emph{ma} is acceptable, while $p$-\emph{ma} is not:

\ea\label{qing} B leaves A's  windowless  room carrying a raincoat. When B returns, A notices that B's raincoat is dry.
\begin{xlist}
\exi{A:}{\gll Mei xia yu  ma? \\
\textsc{neg} fall rain ma \\
\glt `Did it not rain?' \jambox*{($-$\textsc{maq})}}
\exi{A$'$:}[\#]{\gll Xia yu le ma?\\
fall rain \textsc{perf} ma \\
\glt `Did it rain?'\jambox*{($+$\textsc{maq})}}
\end{xlist}
\z

Furthermore, the felicity condition of $-$\textsc{maq} \xref{qn2} accounts for the availability of   the negative \textsc{maq} in \xref{notdrink}, which is translated from the English  example used by \citet{Romero} to show that English low negative questions can convey the speaker's epistemic neutrality towards answers.

\ea\label{notdrink}The speaker is organizing a party and she is in charge
of supplying all the non-alcoholic beverages for teetotalers.
The speaker is going through a list of people that are invited.
She has no previous belief or expectation about their drinking
habits.
\begin{xlist}
\exi{A:} \gll Jane he Mary bu hejiu. \\
Jane and Mary \textsc{neg} drink-alchol \\
\glt  `Jane and Mary do not drink.'
 \exi{S:} \gll Haode. Bill ne? Ta (ye) bu hejiu ma?\\
 good-\textsc{attr} Bill ne? \textsc{3sg} too \textsc{neg} drink-alchol ma \\
\glt `OK. What about Bill? Does he not drink (either)?' \jambox*{($-$\textsc{maq})}
\end{xlist}
\z

In \xref{notdrink}, the goal of the discourse is to `[supply] non-alcoholic beverages' and A has asserted `\textit{Jane and Mary do not drink}', thus we can infer that the current question under discussion is a negative \textit{wh}-question `Who does not drink?'. This means that what is on the Table is the issue, $ \{\neg \mathrm{drink}(\mathrm{j}), \neg \mathrm{drink}(\mathrm{m}), \neg \mathrm{drink}(\mathrm{b})\} $.  Then, A's assertion pushes $\neg \mathrm{drink}(\mathrm{j})$ and $\neg \mathrm{drink}(\mathrm{m})$ onto the Table.  Therefore, $\bigcup_{x=1}^{3}{T}({c})[x]$ = ${T}({c})[1] $ $\cup$ $ {T}({c})[2] $ $\cup$ $ {T}({c})[3] $ = $\{\neg\mathrm{drink}(\mathrm{m})\} $ $\cup$ $\{\neg \mathrm{drink}(\mathrm{j})\}$ $\cup$ \linebreak $\{\neg\mathrm{drink}(\mathrm{j}),$ $\neg \mathrm{drink}(\mathrm{m}),$ $\neg \mathrm{drink}(\mathrm{b}),$ $...\} $ = $ \{\neg \mathrm{drink}(\mathrm{j}),$ $\neg \mathrm{drink}(\mathrm{m}),$ $\neg \mathrm{drink}(\mathrm{b}), ...\} $.\footnote{We could also treat  S's utterance of \emph{Haode} `OK' as an acceptance of A's assertion, thus  $\neg \mathrm{drink}(\mathrm{j})$ and $\neg \mathrm{drink}(\mathrm{m})$ may be already removed from the Table.}   The resulting Table contains $\neg \mathrm{drink}(\mathrm{b})$ but not  $\mathrm{drink}(\mathrm{b})$.  Thus, the context is biased towards $\neg \mathrm{drink}(\mathrm{b})$, even though the speaker does not have any epistemic bias.  Since our definition only requires the context, not the spaker, to be biased, it correctly predicts  the use of  the negative \textsc{maq} in \xref{notdrink} to be felicitous.





In contrast, a negative \textsc{maq} is infelicitous in \xref{posd}, where the goal of the conversation is now to find out who drinks.  The context is biased towards $p$ rather than $\neg p$, thus a $-$\textsc{maq} cannot be used.


\ea \label{posd}The speaker is organizing a party and she is in charge
of supplying all the alcoholic beverages for (alcoholic) drinkers.
The speaker is going through a list of people that are invited.
She has no previous belief or expectation about their drinking
habits.
\begin{xlist}
\exi{S:}[\#]{\gll John bu hejiu ma?\\
John \textsc{neg} drink-alchol ma\\
\glt ` Does John not drink?' \jambox*{($-$\textsc{maq})}}
\end{xlist}
\z
 
 
 
 
In summary, we explain the complementary distribution of positive and negative  \textsc{maq}s summarized in \tabref{maqs} in terms of pragmatic competition.

\begin{table}
\begin{tabularx}{\textwidth}{lCCc}
\lsptoprule
	&  neutral  & biased towards  $p$ & biased towards $\neg p$\\
\midrule
	positive \textsc{maq}s & \cmark & \cmark & \#\\
	negative \textsc{maq}s & \# & \# & \cmark\\
\lspbottomrule
\end{tabularx}
\caption{Distribution of positive and negative \textsc{maq}s}
\label{maqs}
\end{table}

The felicity condition of $-$\textsc{maq}s \xref{qn2} plays a crucial role. A $-$\textsc{maq} has a more specific condition that the context has to be negatively biased.  Thus, whenever this rule applies,   $-$\textsc{maq}s win over +\textsc{maq}s, which are uttered elsewhere, i.e., in neutral and positively biased contexts.   We do not need to stipulate any contextual requirement for +\textsc{maq}s, which are default polar questions. Note also that our definition allows us to uniformly deal with  bias arising from default assertions, contextual compelling evidence and possibility claims.

\subsection{A-not-A questions}

Let us finally turn to \textsc{anaq}s.   As summarized in \tabref{tabanaq},  \textsc{anaq}s are only available in neutral contexts.

\begin{table}
	\begin{tabularx}{\textwidth}{lCCc}
	\lsptoprule
		&  neutral  & biased towards  $p$ & biased towards $\neg p$\\
		\midrule
		\textsc{anaq}s & \cmark & \# & \#\\
	\lspbottomrule
	\end{tabularx}
	\caption{\textsc{anaq}s in neutral contexts}
\label{tabanaq}
\end{table}


In \sectref{subsec:tab}, we define contextual neutrality as in \xref{def:conneut2}.  The context is neutral with respect to $p$ when the Table is empty or the Table contains the issue $\{p, \neg p\}$.


\ea\label{def:conneut2} The context \textsc{c} is neutral with respect to $p$ iff $\bigcup_{x=1}^{n}\mathrm{T}(c)[x]  = \emptyset$ or $\{p, \neg p\}\subseteq\bigcup_{x=1}^{n}\mathrm{T}(c)[x]$,\\
 where $n=|T(c)|$.
\z

Now as discussed in \sectref{subsec:semanaq}, an \textsc{anaq} is always uttered with the boundary tone  $\downarrow$/L\%, which denotes that all the issues on the Table amount to the Hamblin set denoted by the A-not-A construction or that an empty set:

\ea\label{dl2} Semantics of $\downarrow$\\
$\llbracket\downarrow\rrbracket = \lambda \varphi. \lambda {c}. \bigcup_{x=1}^n{T}({c})[x]  = \varphi$ or $\emptyset$,\\
where $n=|T({c})|$
\z


As can be seen from \xref{def:conneut2} and \xref{dl2}, the presence of $\downarrow$/L\% is the source of the neutrality requirement of \textsc{anaq}s.  The intonational morpheme, an exhaustive operator, semantically marks that the context is neutral.


Let us look at specific examples starting with neutral contexts.  The context can be neutral in two ways.  First, an out-of-the-blue context like \xref{outofb} is a neutral context, i.e., the Table is empty ($\bigcup_{x=1}^{1}{T}({c})[{x}] = \emptyset$):
	
\ea\label{outofb} A researcher uses a questionnaire to investigate the relationship between the weather and people's mental states. 
\begin{xlist}
\exi{Q: } \gll Ni de chengshi zuotian  xia mei xia yu?$\downarrow$ \\
you \textsc{gen} city yesterday fall \textsc{neg} fall rain \\
\glt `Did it rain or not rain  yesterday in your city?'\jambox*{(\textsc{anaq})}
\end{xlist}
\z
	
	
Second, the context is neutral when both issues, $\{p\}$ and $\{\neg p\}$, are on the Table.  In \xref{neutral2}, A and B's assertions push $p$ and $\neg p$ onto the Table, respectively.  Thus, at the context after B's assertion, the Table contains both issues ($\bigcup_{x=0}^{2} {T}({c})[{x}] = \{p\}\cup\{\neg p\}=\{p, \neg p\}$).  Thus, according to \xref{def:conneut2}, the context is neutral and compatible with the semantics of $\downarrow$.\footnote{One may wonder whether it is better to separate unmarked \textsc{anaq}s from ones marked with the adverb \emph{daodi} as the one in \xref{neutral2} since the $\{p, \neg p\}$ part in the definition of contextual neutrality in  \xref{def:conneut2} seems to be needed only for \emph{daodi} \textsc{anaq}s.  However, providing independent definitions for unmarked \textsc{anaq}s and \emph{daodi}  questions would not only fail to capture the apparent overlaps in their syntactic structures and   meanings but also such definitions would be inconsistent with each other. Suppose that the neutrality requirement for unmarked \textsc{anaq}s is only that there be no issues on the Table.  On the other hand, the adverb \textit{daodi} presupposes that the question it attaches to is an old question (i.e., the question has already been pushed onto the Table but not solved), so the speaker uses  \textit{daodi} questions to urge the addressee to provide the answer immediately, yielding what \citet{Biezma} calls the cornering effect.  As can be seen, the composition would result in contradiction of the two presuppositions:  The A-not-A construction presupposes that there is no issue while \textit{daodi} presupposes that the issue denoted by the prejacent is already on the Table.  We thus consider both having no issues, $\emptyset$, and having a polar issue, $\{p, \neg p\}$, as cases of contextual neutrality.}
	
\ea\label{neutral2}
\begin{xlist}
\exi{A:} Zuowan xia yu le.
\glt `It rained last night.'
\exi{B:} Bu, meiyou xia. 
\glt `No, it did not rain.'
\exi{C:} \gll (Suoyi / Daodi) xia mei xia yu?$\downarrow$\\
so {} after.all fall not fall rain\\
\glt `(So/After all,) Did it rain or not rain?' \jambox*{(\textsc{anaq})}
\end{xlist}
\z
	
	
When the context is biased toward $p$ by an assertion of $p$ as in \xref{posbana}, the Table contains $\{p\}$, i.e., $\bigcup_{x=0}^{1} \mathrm{T}(c)[{x}] = \{p\}$, but not $\{\neg p\}$.  Therefore, the context contradicts the semantics of $\downarrow$.

\ea\label{posbana}
\begin{xlist}
\exi{A:}[]{Zuowan xia yu le.
\glt `It rained last night.'}
\exi{B:}[\#]{Xia mei xia yu?$\downarrow$
\glt `Did it rain or not rain?' \jambox*{(\textsc{anaq})}}
\end{xlist}
\z


Similarly, when someone asserts $\neg p$, the context is biased towards $\neg p$ as in \xref{meiana}.  Then, the Table contains $\{\neg p\}$, but not $\{p\}$, and becomes incompatible with the semantics of L\%.

\ea\label{meiana}
\begin{xlist}
\exi{A:}[]{Zuowan mei xia yu. 
\glt `It did not rain last night.'}
\exi{B:}[\#]{Xia mei xia yu?$\downarrow$
\glt `Did it rain or not rain?' \jambox*{(\textsc{anaq})}} 
\end{xlist}
\z


	
	
In all other positively and negatively biased contexts discussed in \sectref{subsubsec:pos} and \sectref{subsubsec:neg}, respectively, an \textsc{anaq} is infelicitous. The same explanation applies:  The Table contains only either $\{p\}$ or $\{\neg p\}$ and not the issue with the opposite polarity, which conflicts with the semantics of $\downarrow$.\footnote{An anonymous reviewer points out that the utterance of \emph{maybe} $p$, which is supposed to give rise to a positively biased context as seen in \xref{keneng}, can be followed by  the phrase \emph{Shuo qingchu dian!}  `Please be clear!' and a \emph{daodi} \textsc{anaq}:


\ea\label{clear}
\begin{xlist}
 \exi{A:} \gll Zuowan keneng xiayu-le.\\
 last.night possible rain-\textsc{asp}\\
 \glt `Maybe it rained last night.'
\exi{B:} \gll (Shuo qingchu dian!) Zuowan (daodi) xia mei xiayu?\\
 say clear a.bit last.night after.all rain not rain\\
\glt `(Please be clear!) Did it rain last night or not?'
\end{xlist}
\z


We speculate that the issue that the utterance of \emph{maybe} $p$ raises is $\{p\}$ by default but it could be  $\{p,  \neg p\}$.  Our speculation is motivated by the fact that the presence of the phrase \emph{Shuo qingchu dian!}  `Please be clear!' and the adverb \emph{daodi} in \xref{clear} are crucial since B's utterance of \textsc{anaq}  becomes infelicitous without them:

\ea\label{clear2}
\begin{xlist}
\exi{A:}[]{Zuowan keneng xiayu-le.
\glt `Maybe it rained last night'}
 \exi{B:}[\#]{Zuowan xia mei xiayu?
\glt `Did it rain last night or not?'}
\end{xlist}
\z


The contrast between \xref{clear} and \xref{clear2} shows that \emph{Shuo qingchu dian!} and \emph{daodi} presuppose that there is an unsolved issue $\{p, \neg p\}$ on the Table, though the presupposition is not explicitly spelled out in \xref{clear}.   Thus, the use of  \emph{Shuo qingchu dian!} and \emph{daodi} coerces the contextual update of uttering `\emph{maybe} $p$' from pushing $\{p\}$ onto the Table to pushing   $\{p, \neg p\}$.  This coercion is reasonable given the semantics of existential modal in awareness semantics.   The precise semantics of \emph{maybe}/\emph{keneng} is beyond the scope of the paper, but in awareness semantics \citep{Crone2018,BledinRawlins}, `\emph{maybe} $p$' translates to `an agent is aware of $p$'.  Furthermore, if an agent is aware of  $p$, she is also aware of $\neg p$.}

In short, an \textsc{anaq} can be uttered only in neutral contexts because the intonational morpheme $\downarrow$ that is obligatorily present in the \textsc{anaq} semantically expresses that the context is neutral.
	
\section{Concluding remarks}
The differences among the three kinds of polar questions, +\textsc{maq}s, $-$\textsc{maq}s and \textsc{anaq}s, discussed in this paper are very subtle.  As argued by \citet{YuanHaraGlowinAsia2019}, they all create a Hamblin issue and push it onto the conversation Table.  Previous researchers were also aware that these questions convey different bias meanings, and attempted to characterize the semantics of these questions in terms of the speaker's bias. In this paper, we show that the speaker's bias is not suitable to account for their distribution. Instead, we argue that the contextual bias determines the landscape of Mandarin polar questions.  The notion of the contextual bias is formalized in terms of subjective probability and \quotecite{FB} Table model. The context is biased towards $p$ when it is a common belief that some individual entertains the possibility of $p$ and there is no individual that publicly entertains the possibility of $\neg p$.  Our formalization can correctly predict not only the  biases that arise from the previous assertions in the discourse but also the ones  that arise from non-verbal, contextually-compelling evidence, low-possibility claims, and reported assertions made by non-participants.

Our analysis also has important theoretical implications in the interfaces a\-mong prosody, semantics and pragmatics.  First, by employing the elsewhere condition to explain the division of labor of +/$-$\textsc{maq}s, we can maintain a simple semantics for -\textsc{maq}s and there is no need to stipulate felicity conditions for +\textsc{maq}.  Second, the current analysis supports the idea that a prosodic contour such as $\downarrow$/L\% is an intonational morpheme that can bear semantic content that affects the grammaticality of the construction.


\section*{Acknowledgements}
The research presented in this paper is partly supported by JSPS Kiban (C) \textit{Se\-man\-tic-Pragmatic Interfaces at Left Periphery: A Neuroscientific Approach} (1\-8\-K\-0\-0\-5\-8\-9) awarded to the first author and by MOE (Ministry of Education in China) Project of Humanities and Social Sciences (Project Number: 19YJC740113) a\-ward\-ed to the second author.  We would like to thank the organizers and the audience at the workshop \textit{Biased Questions: Experimental Results \& Theoretical Modelling}  (ZAS) February 4–5, 2021.  All remaining errors are ours.

\printbibliography[heading=subbibliography,notkeyword=this]

\end{document}
