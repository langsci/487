\documentclass[output=paper,colorlinks,citecolor=brown]{langscibook}
\ChapterDOI{10.5281/zenodo.17158196}
\author{Rebecca Woods\orcid{0000-0002-7116-9191}\affiliation{Newcastle University} and Tom Roeper\affiliation{University of Massachusetts Amherst}}
\title[Children's acquisition of English ``high" negation]{Children's acquisition of English ``high" negation: A window into the logic and composition of bias in questions}
\abstract{In our investigation of biased ``high'' negation questions (NegQs) and negative tag structures, we present production data from early child acquisition of English and judgements from adult English. We use this data to demonstrate that the structures of negative tags and NegQs are distinct despite similarities in their interpretation, spelling out how distinct structures lead to differences in use and acquisition. We also highlight the remarkable ability of very young children to manipulate a discourse context shared with another person using increasingly fine-grained syntactic structures.}

%move the following commands to the "local..." files of the master project when integrating this chapter

\IfFileExists{../localcommands.tex}{
   \addbibresource{../localbibliography.bib}
   \usepackage{tabularx,multicol}
%	\setlength{\multicolsep}{6.0pt plus 2.0pt minus 2.0pt}
\usepackage{array} % for the 'm' column type
\usepackage{multirow}

%font:
\usepackage{siunitx}
\sisetup{group-digits=none}

\usepackage{textcomp} %emdash

%\usepackage{libertinus−otf}
%\setmainfont{Libertinus}

 
\usepackage{langsci-optional}
\usepackage{langsci-lgr}
\usepackage{langsci-gb4e}
\usepackage{langsci-basic}
\usepackage{langsci-affiliations}
\usepackage{langsci-branding}

\usepackage{url}
\urlstyle{same}
\usepackage{orcidlink}

%\usepackage{langsci-textipa}

\usepackage{amsmath}
\usepackage{amssymb}
\usepackage{stmaryrd}

%\usepackage{biblatex}%citation input! DO NOT CHANGE!
%\usepackage[american]{babel}
%\usepackage{csquotes}
%\usepackage[style=apa]{biblatex}
%\bibliographystyle{linquiry2}
%\usepackage[hidelinks, bookmarks=false, pdfstartview=FitH]{hyperref} %bookmarks=false, urlcolor=blue,

\usepackage{pifont} %checkmarks
%\usepackage{ulem}

%no new packages for 02.tex :

%\usepackage{setspace}
%\doublespacing
%\singlespacing


\usepackage{tikz-qtree}
\usepackage{tikz-qtree-compat}
\tikzset{every tree node/.style={align=center,anchor=north}}
%\usepackage{qtree,tree-dvips}%for trees, dvips won't work with figures unless figures are converted to .eps. make sure Typeset is set to "tex and DVI", not "pdftex" 
%\qtreecentertrue
\usepackage[linguistics]{forest}
\forestset{
fairly nice empty nodes/.style={
delay={where content={}
{shape=coordinate, for siblings={anchor=north}}{}},
for tree={s sep=4mm} }
            }


\usepackage[nameinlink]{cleveref} %for \Cref{}
% \usepackage{comment}
% \usepackage{color}
% \usepackage{subcaption}
\usepackage{subfigure}
% \usepackage{caption}
\usepackage{arydshln}

% \usepackage[scale=0.8]{FiraMono}

%\usepackage[export]{adjustbox}

%03.tex packages:
%\usepackage{linguex}%for examples

%04.tex packages:
\usepackage{cancel}
\usepackage{metre}
% \pagenumbering{roman}

%08.tex packages:
% \usepackage{xeCJK} %Chinese fonts
% \newfontfamily{\NotoSerifTC}{Noto Serif TC}
% \setCJKmainfont{Noto Serif TC}

   %for all .tex files (Affiliation setups):
\SetupAffiliations{ output in groups = false,
separator between two = {\bigskip\\},
separator between multiple = {\bigskip\\},
separator between final two = {\bigskip\\},
orcid placement=after
}

% ORCIDs in langsci-affiliations 
\definecolor{orcidlogocol}{cmyk}{0,0,0,1}
\RenewDocumentCommand{\LinkToORCIDinAffiliations}{ +m }
  {%
    \,\orcidlink{#1}%
  }

\makeatletter
\let\thetitle\@title
\let\theauthor\@author
\makeatother

\newcommand{\togglepaper}[1][0]{
   \bibliography{../localbibliography}
   \papernote{\scriptsize\normalfont
     \theauthor.
     \titleTemp.
     To appear in:
     E. Di Tor \& Herr Rausgeberin (ed.).
     Booktitle in localcommands.tex.
     Berlin: Language Science Press. [preliminary page numbering]
   }
   \pagenumbering{roman}
   \setcounter{chapter}{#1}
   \addtocounter{chapter}{-1}
}

\newbool{bookcompile}
\booltrue{bookcompile}
\newcommand{\bookorchapter}[2]{\ifbool{bookcompile}{#1}{#2}}

\newcommand{\cmark}{\ding{51}}%
\newcommand{\xmark}{\ding{55}}%

%for 01.tex:
\newcommand{\eval}[2]{\llbracket #1\rrbracket^{#2}}

%02.tex:
\newcommand{\smiley}{:)}

%03.tex:
\newcommand{\exa}{\ea}
%\renewcommand{\firstrefdash}{}%changes citations from, e.g., (2-a) to (2a)
\newcommand{\den}[2]{\ensuremath{\llbracket#1\rrbracket}\textsuperscript{\ensuremath{#2}}}
\newcommand{\citepos}[1]{\citeauthor{#1}'s (\citeyear{#1})}
\newcommand{\tp}[1]{\ensuremath{{\langle #1 \rangle}}}

\newcommand{\rise}{$\nearrow$\xspace}
\newcommand{\fall}{$\searrow$\xspace}
\newcommand{\notp}{\emph{not}-$p$\xspace}

%09.tex:
%\newcommand\den[1]{\ensuremath{[\![ #1 ]\!]}}
%\newcommand{\int}[1]{\ensuremath{\llbracket #1 \rrbracket}}

%13.tex:
\newcommand{\PreserveBackslash}[1]{\let\temp=\\#1\let\\=\temp}
% \newcolumntype{C}[1]{>{\PreserveBackslash\centering}p{#1}}
\newcolumntype{R}[1]{>{\PreserveBackslash\raggedleft}p{#1}}
\newcolumntype{L}[1]{>{\PreserveBackslash\raggedright}p{#1}}

\newcommand{\SB}{\textsubscript}
\newcommand{\SuB}{\textsuperscript}

\newcommand{\quotecite}[1]{\citeauthor{#1}'s (\citeyear*{#1})}

   %% hyphenation points for line breaks
%% Normally, automatic hyphenation in LaTeX is very good
%% If a word is mis-hyphenated, add it to this file
%%
%% add information to TeX file before \begin{document} with:
%% %% hyphenation points for line breaks
%% Normally, automatic hyphenation in LaTeX is very good
%% If a word is mis-hyphenated, add it to this file
%%
%% add information to TeX file before \begin{document} with:
%% %% hyphenation points for line breaks
%% Normally, automatic hyphenation in LaTeX is very good
%% If a word is mis-hyphenated, add it to this file
%%
%% add information to TeX file before \begin{document} with:
%% \include{localhyphenation}
\hyphenation{
    par-a-digm
}
\hyphenation{
que-stions
}
\hyphenation{
na-me-l-y
}
\hyphenation{
ge-ne-ra-tion
}
\hyphenation{
Hir-sch-berg}
\hyphenation{
stee-p-er
}
\hyphenation{
inter-ro-ga-tives
}
\hyphenation{
cons-truc-tion
}
\hyphenation{
p-u-sh-ed
}
\hyphenation{
A-mong
}
\hyphenation{
award-ed
}
\hyphenation{
synta-ctic
}
%\hyphenation{
%wh-ich
%}
\hyphenation{
call-ed
}
\hyphenation{
mo-no-po-lar
}
\hyphenation{
proso-dic
}
\hyphenation{
non-ve-ri-di-cal
}
\hyphenation{
Ro-me-ro
}
\hyphenation{
though
}
\hyphenation{
ra-ther
}
\hyphenation{
mo-da-li-ty
}
\hyphenation{
prag-ma-ti-cal-ly
}
\hyphenation{
trans-ver-sal
}
\hyphenation{
re-se-arch
}
\hyphenation{
clau-s-es
}
\hyphenation{
c-lau-se
}
\hyphenation{
spea-k-er
}
\hyphenation{
a-mon-g-st
}
\hyphenation{
th-rou-gh
}
\hyphenation{
ad-dres-see
}
\hyphenation{
mo-da-li-s-ed
}
\hyphenation{
Ja-mie-son}








\hyphenation{
    par-a-digm
}
\hyphenation{
que-stions
}
\hyphenation{
na-me-l-y
}
\hyphenation{
ge-ne-ra-tion
}
\hyphenation{
Hir-sch-berg}
\hyphenation{
stee-p-er
}
\hyphenation{
inter-ro-ga-tives
}
\hyphenation{
cons-truc-tion
}
\hyphenation{
p-u-sh-ed
}
\hyphenation{
A-mong
}
\hyphenation{
award-ed
}
\hyphenation{
synta-ctic
}
%\hyphenation{
%wh-ich
%}
\hyphenation{
call-ed
}
\hyphenation{
mo-no-po-lar
}
\hyphenation{
proso-dic
}
\hyphenation{
non-ve-ri-di-cal
}
\hyphenation{
Ro-me-ro
}
\hyphenation{
though
}
\hyphenation{
ra-ther
}
\hyphenation{
mo-da-li-ty
}
\hyphenation{
prag-ma-ti-cal-ly
}
\hyphenation{
trans-ver-sal
}
\hyphenation{
re-se-arch
}
\hyphenation{
clau-s-es
}
\hyphenation{
c-lau-se
}
\hyphenation{
spea-k-er
}
\hyphenation{
a-mon-g-st
}
\hyphenation{
th-rou-gh
}
\hyphenation{
ad-dres-see
}
\hyphenation{
mo-da-li-s-ed
}
\hyphenation{
Ja-mie-son}








\hyphenation{
    par-a-digm
}
\hyphenation{
que-stions
}
\hyphenation{
na-me-l-y
}
\hyphenation{
ge-ne-ra-tion
}
\hyphenation{
Hir-sch-berg}
\hyphenation{
stee-p-er
}
\hyphenation{
inter-ro-ga-tives
}
\hyphenation{
cons-truc-tion
}
\hyphenation{
p-u-sh-ed
}
\hyphenation{
A-mong
}
\hyphenation{
award-ed
}
\hyphenation{
synta-ctic
}
%\hyphenation{
%wh-ich
%}
\hyphenation{
call-ed
}
\hyphenation{
mo-no-po-lar
}
\hyphenation{
proso-dic
}
\hyphenation{
non-ve-ri-di-cal
}
\hyphenation{
Ro-me-ro
}
\hyphenation{
though
}
\hyphenation{
ra-ther
}
\hyphenation{
mo-da-li-ty
}
\hyphenation{
prag-ma-ti-cal-ly
}
\hyphenation{
trans-ver-sal
}
\hyphenation{
re-se-arch
}
\hyphenation{
clau-s-es
}
\hyphenation{
c-lau-se
}
\hyphenation{
spea-k-er
}
\hyphenation{
a-mon-g-st
}
\hyphenation{
th-rou-gh
}
\hyphenation{
ad-dres-see
}
\hyphenation{
mo-da-li-s-ed
}
\hyphenation{
Ja-mie-son}








   \boolfalse{bookcompile}
   \togglepaper[23]%%chapternumber
}{}

\begin{document}
\maketitle

\section{Introduction}
In this chapter we argue that data from non-canonical biased questions in early language acquisition can greatly enhance and refine our formal understanding of such questions and how we conceptualise the integration of propositional and contextual information in the minds of language users. Children's production has not, to our knowledge, been considered in theoretical accounts of biased structures and their meanings; nor have biased structures received much attention in existing acquisition literature, except from the point of view of non-target structures in child syntax (e.g. \citealt{guastietal1995}). We therefore make an empirical contribution by focusing on English-acquiring children's production of ``high'' negation structures in naturalistic speech settings. We also make a theoretical contribution by revealing structural differences between NegQs and negative non-matching\footnote{This means that the polarity of the tag and its associated declarative clause (its anchor) do not match. In the context of our chapter, all anchors have positive polarity.} tag questions on the basis of this data, as well as claiming a structural distinction between rising and falling negative tag questions. 

Let us first present the core data for this chapter. \xref{intro:negqs} illustrates negative polar questions (NegQs).\footnote{As some NegQs, like (\ref{intro:negqa}), are string-equivalent to negative polar exclamatives, we hand-checked all potentially ambiguous strings in context to determine whether the string was used as a NegQ. See \sectref{sect:quantdata} for more details.} All examples in \xref{intro:negqs} are taken from the CHILDES database \citep{macwhin2000}; corpus names are given in parentheses.

\ea \label{intro:negqs}
	\ea Isn't that funny? \label{intro:negqa}\phantom{a} \hfill Child 2 (Valian), 1;9
	\ex There, don't you see it? \phantom{a} \hfill Ross (MacWhinney), 2;4
	\ex Doesn't it feel good, ma?\phantom{a} \hfill Victor (Gleason), 2;4 
	\ex Isn't this mine? Isn't this mine? \phantom{a} \hfill Barbara (Belfast), 2;7
 \z
\z

\il{US English}
\il{Belfast English}

In NegQs, the clitic negation \textit{n't} is, we claim, not propositional but metalinguistic (see \citealt{goodhue2022allc, goodhue2022nls} for a similar claim, building on intuitions by \citealt{ladd1981}). This metalinguistic negation is structurally and semantically distinct from the negation we see in superficially similar negative tag structures like \xref{intro:tagqs}. In \xref{intro:negqs}, metalinguistic negation scopes above the propositional content of the utterance, negating the typical interpretation of an interrogative-typed clause, namely, that the speaker is ignorant as to the truth of the proposition and expects their addressee to be knowledgeable.\footnote{Details to follow in \sectref{sect:metalingneg} and \sectref{sect:negresponsepatterns}.}  In \xref{intro:tagqs}, \textit{n't} is interpreted as propositional negation that scopes under an interrogative clause-type operator, negating some proposition \textit{p} within that clause. 

\ea \label{intro:tagqs}
	\ea Close to Rachel's feet, wasn't it? \label{intro:tagqa} \phantom{a} \hfill Anne (Manchester), 1;11
	\ex Now it needs ironing, doesn't it? \phantom{a} \hfill Gail (Manchester), 2;3
	\ex We saw some at the zoo, didn't we?\phantom{a} \hfill Joel (Manchester), 2;6
 \z
\z

\il{British English}

The structures in \xref{intro:tagqs} consist of positive anchors -- affirmative declarative clauses that first introduce the proposition at issue -- and negative tags, the structure of which is a key proposal in this chapter.\footnote{We do not consider matching tags -- i.e. negative anchors with negative tags -- in this chapter, though we do find a very small number of them in the dataset.} We do not have prosodic information for these structures, as the original audio is no longer available, but following \citeauthor{dehebraun2013}'s (2013) work on the British component of the International Corpus of English, we assume that they, like most reverse-polarity tags in adult English, are likely composed of two intonational contours; one for the anchor and one for the tag. Such tags are often referred to in the literature as nuclear negative tag structures (following \citealt{ladd1981}). The negation that they contain is, we argue, typical propositional negation.

We will propose structures for NegQs and negative tag questions assuming a speech act syntactic framework as in \xref{intro:structure}, which we assume is present in the left periphery of all root utterances.

\ea \label{intro:structure}
[SpeechActP [PerspectiveP [CP [TP….]]]] \jambox*{cf. \citet{woods2021please}}
\z

This structure contains two discourse-related syntactic projections, Speech\-ActP and PerspectiveP. SpeechActP, the highest syntactic node in \xref{intro:structure} (and therefore in \xxref{intro:negqs}{intro:tagqs}), contains a discourse commitment operator. All root utterances contain one of these operators, which mark to what and to whom the speaker\footnote{We recognise here that terminology such as ``speaker" and ``speech act" can be exclusionary and ableist. We use the terms ``speaker" and ``speech act" throughout because our primary empirical data is spoken child English, but we believe that our theoretical assumptions and claims about interlocutor-information relationships can apply whether or not the interlocutors in question use spoken or signed languages.} is committed. This is very like the concept of the Common Ground as articulated by \citet{stalnaker1979, stalnaker2002, gunlogson2001, gunlogson2008} and others, but the commitment-based approach we use here (for further elaboration see e.g. \citealt{krifka2015SALT,geurts2019}) differs from intentionalist, belief-oriented approaches to speech acts, so we will briefly explain what discourse commitment operators express. In this chapter we focus on operators that express one of two types of commitment between the speaker, the addressee, the propositional content and the discourse context. One type is found in utterances, in the making of which a speaker commits in the discourse context to acting as though some proposition is true (essentially, assertions).\footnote{In \citet{guroeper2011} and \citet{Roeper2016}, a slightly different perspective is advanced about implicit arguments, general point of view and what the affirmation of a speech act involves. Of course, a dialogues carries implications and invokes inferences beyond calculation of Common Ground or commitments and the pursuit of assent or confirmation.}
%\footnote{It is plausibly true that some propositions could be expressed as being general truths to which all interlocutors (indeed, all language users) are considered to be committed (e.g. \citealt{guroeper2011} on general Point of View in child language). We will not deal with `statements' of this kind in this chapter as the data included here all deal with instances where speakers are committed to achieving particular discourse-specific goals in conjunction with specific interlocutors.} 
This will be referred to in shorthand as \textit{the speaker commits to the proposition}. The second type is found in utterances, in the making of which the speaker commits to the realisation of a goal, namely the addressee committing to acting as though some proposition is true (essentially, questions).\footnote{Note that there are issues here around pragmatic recursion or recursion of interlocutor goals, which could have interesting consequences for theories of pragmatic development and evolution of communication. We do not have space to deal with these issues here and leave them for future work.} This will be referred to in shorthand as \textit{the speaker commits to resolving the issue of some proposition}. These operators, \textsc{assert} and \textsc{question}, are often realised in English as different intonation contours on the right edge of the utterance, so for our purposes, on the tag part of negative tag structures.\footnote{See \citet{Heim2019} for an approach to mapping prosody and speech acts that does not assume speech act operators, but demonstrates a relationship between pitch excursion and commitment; namely that rising pitch contours correlate with reduced speaker commitment to the proposition.} 

These operators are not unique to NegQs or tag structures, but these non-canonical question structures are excellent proving grounds for their presence and effects. We will claim that NegQs are fundamentally \textsc{question} structures containing interrogative clauses that generate the commitments typical of questions – i.e. that the speaker is committed to the goal of resolving an issue, via an expectation that the addressee will provide an answer. Note that this implies that NegQs will typically be used when the speaker believes this expectation will be met; an assumption made by speech act theorists from \citet{searle1969} to \citet{farkas2022} and many in between. 

Negative tag structures, on the other hand, are split into two types. \textsc{assert} negative tag structures are fundamentally assertions and behave as such in discourse, while \textsc{question} negative tag structures are fundamentally questions in their discourse commitments.

We move down the structure in \xref{intro:structure}, now, to the second discourse-related projection, PerspectiveP, which encodes interlocutor perspective. That is, elements in PerspectiveP mark and modify from whose perspective we should understand the propositional
content of the CP. The range of operators that can merge in PerspectiveP is much larger, ranging from representations of the speaker and addressee, to modal and logical operators. In the case of NegQs, we will argue that metalinguistic negation is merged in PerspectiveP; for negative tag structures, we argue for representations of the speaker and addressee. 

With these two projections, SpeechActP and PerspectiveP, and the elements that merge in them, we can make specific claims about the source of bias in NegQs and tag structures; that is to say, how the speaker expresses their prior knowledge or beliefs while still looking to elicit a response from their addressee. We will claim that the bias in a NegQ arises from the metalinguistic negation and its interaction with interrogative clause typing. In \textsc{assert} negative tag structures, \textit{bias} arises from the fact that the user actually does assert that a particular proposition is true from their perspective. In \textsc{question} negative tag structures, bias arises through the interaction of the anchor proposition and an \textsc{addressee} operator in PerspectiveP (details to follow in \sectref{sect:clausetypesnotpersps}). We will demonstrate that these proposals predict, correctly, that the bias in \textsc{question} tag structures is more similar to that in NegQs that to \textsc{assert} tag structures, without being identical or derived from NegQ bias.

In addition to our syntactic claims, we use \quotecite{farkasbruce2010} Table model, as updated by \citet{farkas2022}, to fully model the pragmatic characteristics of these utterances that fall in part out of our syntactic analyses. These frameworks allow us to demonstrate our claims that NegQs and negative tag structures are syntactically different (contra \citealt{sailor2012}, \citealt{jamieson2018}, a.o.) and pragmatically distinct in how they generate and communicate bias. Note that we do not give in this chapter formal semantic denotations for metalinguistic negation or the operators we propose. In this chapter we will demonstrate the meanings of these operators using paraphrase and leave formalisations for future work.

Our account is motivated by empirical evidence from the developmental path of English-acquiring children that we present in this chapter: children acquire and use negative tag structures before they begin to use NegQs. This evidence strongly supports an account of these constructions that is parsimonious in its lexical array in order to explain very early, target-like acquisition of negative tag structures. However, it is fine-grained with respect to the high left-periphery of the clausal structure to allow children to express the complex relationships between interlocutors and propositional material that they develop the ability to conceptualise. We therefore exhort theorists to take account of acquisition data wherever possible rather than relying on adult production and intuitions, not least so that we do not try to reduce to specialised operators that which we can achieve with a compositional, acquirable concept of syntactic structure and its interfaces with other modules.

The chapter is structured as follows. In \sectref{sect:ourdata} we give a quantitative over\-view of how children produce negative tag structures and NegQs before looking deep\-er into the qualities of these utterances. We present our proposal for the syntax of negative tag structures (\sectref{sect:proposal}) and NegQs (\sectref{sect:metalingneg}), then in \sectref{sect:strengths} demonstrate how our proposal captures the child's developmental path. In \sectref{sect:strengths} we also introduce new diagnostics pertaining to response and assent patterns that further support our syntactic claims. We then summarise in \sectref{summary}.

\section{``High'' negation in child English}\label{sect:ourdata}

\subsection{Quantitative data}\label{sect:quantdata}

The use of ``high'' negation in child English follows a child's first use of tense, auxiliaries (sentence-medially and sentence-initially) and fronted wh-elements. However, it is still used quite early in acquisition despite the complexity of the meanings attributed to it and the fact that it requires fronting of the tensed auxiliary and fronting (or base-generation) of clitic negation.

We used the Wang CHILDES browser\footnote{\url{https://naclo.cs.umass.edu/childes-search/}, currently maintained by Christa Bowker. The corpora are pulled from CHILDES (\url{https://childes.talkbank.org/}, \citealt{macwhin2000}). To avoid overlong citations in our in-text examples, please see the appendix for a guide to references for the CHILDES corpora cited in this chapter.} to search 44 UK and US English corpora for ``high'' negation. We searched the CHI tier for instances of BE, DO and HAVE auxiliaries with clitic negation\footnote{\textit{isn't, wasn't, aren't, weren't, don't, didn't, doesn't, haven't, hasn't, hadn't, ain't}.} in children up to age 4, returning over 20,000 hits. We then removed all instances of negative declaratives, utterances with non-overt or indecipherable subjects, song lyrics, and one example of a misattribution of an adult utterance to CHI. We also separated off imitations, wh-questions, negative anchors with positive tags, lone tags without a clear anchor within 5 lines, negative polar questions without inversion, and any unclear structures. 

This left us with 633 instances of true ``high'' negation structures from 67 children across 24 corpora including negative tag structures with positive anchors, negative imperatives with overt subjects, and auxiliary-initial structures containing ``high'' negation. These latter we then tagged for the act that was being performed by the structure: biased polar question (NegQ), negative polar exclamative, or `persuasion' question.\footnote{Persuasion questions are polar questions used to exhort the addressee to do something – they are similar in effect to an imperative. Imagine a parent trying to get out of the house who says to their child ``Can't you just put your shoes on already!" This utterance is neither a question requiring a response, nor an exclamative in the typical, surprise at some exceeded degree, sense. They are termed `suggestion' questions by \citet{romerohan2004}, who also mention them briefly.} We did this by hand using features of individual utterances (e.g. the absence of a gradable predicate predicts that the structure is not an exclamative) and up to 5 lines of discourse preceding and following the utterance to judge the utterance in context. We also included an ``other" category for imperatives with overt subjects, as these are not used as question acts, in addition to instances where other aspects of syntax left us unsure as to the act involved, context suggested a different reading, or context didn't help to differentiate possible readings. Examples of each of the categories are shown below.\footnote{For access to the resulting database of English ``high" negation, please contact the first author.}

\ea TAG: Got got a small boy haven't we Mummy\phantom{a}\hfill Anne (Manchester), 1;11
\ex IMPERATIVE: Don't you pee pee in the big girl pants.\phantom{a}\hfill Eve (Brown), 1;11
\ex BIASED Q: There, don't you see it?\phantom{a}\hfill Ross (MacWhinney), 2;4
\ex EXCLAMATIVE: Isn't it sweet.\phantom{a}\hfill Anne (Manchester), 2;5
\ex PERSUASION: Mommy, don't you think we could play?\phantom{a}\hfill Abe (Kuczaj), 3;4
\ex OTHER: Mommy, isn't this a house or apartment?\phantom{a}\hfill Abe (Kuczaj), 3;6
\z

\il{US English}
\il{British English}

We found that tag structures were by far the most common structures containing high negation in the corpus (457 instances). NegQs were next most common (74), followed by imperatives with overt subjects (38) and negative polar exclamatives (36). There were just 3 examples of `persuasion' questions. Tag structures were also used earliest (38/51 utterances before age 2;6), followed by NegQs (7/51) and exclamatives (1/51). The full breakdown of act by age is shown below, in this table adapted from \citet[765]{woodsroeper2021}:

\begin{table}
  \begin{tabularx}{\textwidth}{lYYrrYY}
  \lsptoprule
            & TagQ & NegQs  & NegExcl & Persuasion & Other & Total \\
  \midrule
  <2;0  &   7  &    2  &    0      & 0 & 1 & 10 \\
  2;0--2;5  &   31  &    5  &   1      & 0 & 4 & 41 \\
  2;6--2;11  &   268  &  25  &    5      & 1 & 31 & 330 \\
  3;0--3;5  &  94  &   15  &    14     & 1 & 12 & 136 \\
  3;6--3;11  &  57  &    27  &   16      & 2 & 14 & 116 \\
  \midrule
  Total &   457  &    74  &    36     & 4 & 62 & 633 \\
  \lspbottomrule
 \end{tabularx}
 \caption{High negation questions by age and act}
\label{tab:highneg:ageact}
\end{table}

Emerging in \tabref{tab:highneg:ageact} is an acquisition path that we aim in the rest of this chapter to capture: tag structures emerge before (and in greater numbers) NegQs, which emerge before negative polar questions and persuasion questions. This holds across children, as illustrated in \tabref{tab:highneg:ageact}, but also within individuals. \tabref{tab:highneg:ageact} also suggests that children acquiring English have a sophisticated and nuanced understanding of different types of negation that interacts with different relationships between interlocutors and the propositional material they are trying to share. This understanding develops and changes over a short space of time.

The rest of this chapter aims to make sense of and account for the first steps in the acquisition path in \tabref{tab:highneg:ageact}. We will focus on unpacking the syntax and pragmatics of negative tag structures and NegQs in child production and in adult English. We first take a qualitative look at the earliest NegQs and negative tag structures in the child data (\sectref{sect:qualdata}). As mentioned above, we will argue that our acquisition data supports an analysis of negative tag structures whereby the tag is not simply an elided NegQ, principally because negative tag structures precede NegQs in acquisition. We provide an analysis for NegQs that combines insights from \citet{krifka2015SALT}, \citet{goodhue2022allc} and \citet{holmberg2016} to capture our data and form the basis for a minimal and plausibly acquirable analysis of ``high'' negation structures (\sectref{sect:ourwork}). %We also briefly present our analyses for negative polar exclamatives and persuasion questions in \sectref{sect:biggerpicture}.

\subsection{Qualitative data}\label{sect:qualdata}

Existing work on negative tag structures and NegQs is clear that sophisticated discourse management skills are required to use and interpret such structures. Negative tags require the user to recognise conventional uses of particular syntactic structures and model the cognitive state, however shallowly, of their interlocutor (see, e.g. \citealt{sadock1974, ladd1981, asherreese2007, reeseasher2008, malsteph2015} a.m.o.). NegQs also require the user to recognise marked uses of syntactic conventions, in this case combining negation and polar interrogatives to express a bias that they hold. The data in \tabref{tab:highneg:ageact}, therefore, demonstrate that children around the age of 2 are already sophisticated conversationalists. Some of them are aware of the possibility that their beliefs are not shared by others and they are capable of expressing this through the choice of linguistic structures that they employ. Take as an example the two NegQs used before age 2, which are used after the child's assumptions are put into doubt by a previous utterance. 

\begin{exe}
\ex 02b.cha (Valian), 1;9 \label{3:funny}
\begin{xlist}
\exi{MOT:} 	did you play marbles with cousin George?
 \exi{CHI:} yeah! [laughs]
 \exi{MOT:} that's funny?  
 \exi{CHI:} \textbf{isn't that funny}?
\end{xlist}
\il{US English}
\end{exe}

\begin{exe}
\ex Joel (Manchester), 1;11 \label{3:dinner}
\begin{xlist}
 \exi{MOT:} tell Caroline what you're gonna have for your dinner.
  \exi{INV:} what are you gonna have for your dinner?  
  \exi{CHI:} \textbf{don't you know}?
\end{xlist}
\end{exe}
\il{British English}

In both cases, the child's interlocutor asks a question that causes them to question some previously held belief; they then ask a NegQ to check whether the propositional content of the belief can still hold. These uses of NegQs chime with adult uses of the same structures.

The earliest negative tag structures in our corpus behave a little differently. Negative tags can be used to request confirmation of a proposition that the speaker believes to be true, as in \xref{tagparkrise}. However, they can also be used when the speaker is more certain of the proposition, but wishes to ``hedge" (in \citeauthor{ladd1981}'s (1981) terms) or seeks only ``acknowledgement'' of the proposition by their addressee (in \quotecite{asherreese2007} terms), as in \xref{tagparkfall}.

\begin{exe}
    \ex Context: A is fairly sure that B wants to go to the park, but they're slow to put their shoes on at the door.
    \sn A: You want to go to the park, don't you?\label{tagparkrise}
    \ex Context: A and B have discussed going out after lunch. At 12.30pm, B is by the front door, shoes on, with bucket and spade in hand.
    \sn A: You want to go the park, don't you.\label{tagparkfall}
\end{exe}

In the earliest part of our dataset, acknowledgement-type uses appear to be more common.\footnote{We identify the uses of tag structures in our dataset on the basis of context alone; none of the corpora below have audio files attached, so prosodic information is not available to us. We do not infer prosodic information from punctuation in the transcript (i.e. the use of turn-final \textit{.} vs \textit{?} to represent intonation) because this is not consistent across corpora. In any case, it is unclear what the prosody of tag structures is when they are produced by children, as this has not yet been studied.} In \xref{3:know}, a child of 1;11 appears to be looking simply to gain her mother's attention using a negative tag structure, in the middle of a period of monologuing (numbers in brackets represent pauses in seconds). In this instance, during a period of toy play, it is thought that knowledge about the smallness of the boy is shared knowledge:

\begin{exe}
\ex Debbie (Wells), 1;11 \label{3:know}
\begin{xlist}
\exi{CHI:}	Gotto pick it up. Throw it out. Pick it up. Throw it out. xxx (14). 
 	Got a boy. Got a got a small boy, \textbf{haven't we} Mummy?
 	We've got a big girl (2). xxx get a big girl. Look Mum I'm nearly getting big.
\exi{MOT:}	You are getting big, mm.
\end{xlist}
\end{exe}
\il{British English}

In this case, both discourse participants appear have the same shared knowledge, and the proposition asserted by the declarative anchor is used as a summary or a verbal recognition of an event in the world, while the negative tag functions to recognise that the other participant is present and knows this too. This could be considered a highly biased use of the negative tag structure as the question part of the structure is barely a question at all. Another such example is found in \xref{3:blocks}, where the child's negative tag structure is uttered at the same time as MOT's second utterance:

\begin{exe}
\ex Emma (Tardif), 1;9 	\label{3:blocks}
\begin{xlist} 
\exi{CHI:}	It's driving
\exi{MOT:}	voom. It ran into the blocks. voom. voom.
\exi{CHI:}	voom. the blocks. blocks. the blocks fell, <\textbf{didn't they}>.
\exi{MOT:}	where is the car going? oops, it's on the floor. 
\end{xlist}
\end{exe}

Similarly, in \xref{3:feet}, a child repeats information she has already given, followed by a tag question that appears to seek acknowledgement that her mother has understood the proposition. She is unlikely to be asking for confirmation from her mother, who has originally requested this information.

\begin{exe}
\ex Anne (Manchester), 1;11 \label{3:feet}
\begin{xlist}
\exi{CHI:} closer
\exi{MOT:} closer? what was it close to?
\exi{CHI:} Rachel.
\exi{MOT:} $[$unintelligible$]$
\exi{CHI:} close to Rachel's feet, \textbf{wasn't it}?
\exi{MOT:} huh?	
\end{xlist}
\end{exe}
\il{British English}

Note that the data in \xxref{3:know}{3:feet} suggest that early negative tag structures are truly generated rather than fixed forms, given that they (a) contain a range of auxiliaries in a range of forms inflected for tense and person, (b) contain subject pronouns of various persons and numbers and that (c), the auxiliary and subject in the tag always match those in the anchor.

Given the quantitative and qualitative child data, we now move on to our proposal for the structure of negative tag structures and NegQs. We diverge from accounts that claim negative tag structures contain NegQs (e.g. \citealt{sailor2012}) on the grounds that if acquisition order reflects complexity of syntactic structure, negative tag structures must be syntactically less complex than NegQs. This is counterintuitive on a surface level, since NegQs are monoclausal while negative tag structures are biclausal. However, we argue for a complex discourse-oriented left periphery that hosts syntactic, prosodic and interpretive cues to the child and that from this point of view, negative tag structures are globally less complex. Specifically, we claim that the discourse-oriented left periphery is a target for movement and affects scope relations; the English-acquiring child must learn for NegQs that sentence-initial metalinguistic negation entails an operation on the discourse-oriented left periphery and not simply on CP. We will also go on to examine adult response patterns to negative tag structures and NegQs to refine our syntactic and pragmatic proposals.

\section{Tag structures, NegQs, and non-canonicity}\label{sect:ourwork}
\subsection{Our proposal}\label{sect:proposal}
\subsubsection{Assumptions}

We assume an extended left periphery where CP is split into three positions. The highest position, SAP (Speech Act Phrase), hosts speech act adverbs and operators that express speaker commitments (or the speaker's expectations for their interlocutor to commit). This scopes over PerspectiveP, which hosts operators that express speaker intentions and point of view. PerspectiveP scopes over CP, which hosts clause typing elements. Thus we assume that clause type and ``illocutionary force" are not automatically linked, in line with \citet{conigzeg2012} a.o. There are parallels between this approach and proposals by \citet{hill2013}, \citet{krifka2021} and \citet{wiltschko2021}; see \citet{woods2021please} for a summary of specific similarities and differences.

\subsubsection{Joining two clauses (a first pass at a negative tag structure proposal)}\label{sect:negtagpass1}

Using the assumptions above, we propose that negative tag structures are embedded in an extended left periphery containing a speech act projection as in \xref{negtag:pass1}, where \textsc{decl} and \textsc{q} are clause-typing, not speech act, operators.

\ea Negative tag structure: first pass\label{negtag:pass1}
\ea Lucy is coming, isn't she?
\ex $[$\SB{SAP} \textsc{operator} $[$\SB{CP} $[$\SB{CP} \textsc{decl} $[$\SB{IP} Lucy is coming $]] [$\SB{C} $\bigwedge ] [$\SB{CP} \textsc{q} isn't $[$\SB{IP} she t\SB{isn't} \sout{coming} $]]]]$\label{negtag1:b}
\z\z

Note that the conjunction in \xref{negtag:pass1} is not of speech acts, but of the two CPs, i.e. typed clauses that have not been specified for a particular speaker perspective or commitment. The decoupling of clause typing and perspective is important but often only implicit in speech act theorising; we explicitly justify this decoupling in \sectref{sect:tagpart}.  There is no contradiction in the conjunction of the typed clauses in \xref{negtag1:b}, as we will demonstrate below.

Negative tag structures can receive either final falling or rising intonation contours (see, e.g. \citealt{dehebraun2013}). For this reason we claim that in \xref{negtag:pass1}, \textsc{operator} may be \textsc{assert} or \textsc{question}. The realisation of \textsc{assert} and \textsc{question} in English are prosodic and contribute to the interpretation of the tag structure. Taking \textsc{assert} first, the tag structure receives falling intonation in the tag part; that is, the rightmost, last pronounced part of the utterance, represented by the final ↘ in \xref{asserttag1:b}. \xref{asserttag1:c} contains a step-by-step paraphrase for each part of the structure in \xref{asserttag1:b}.

\ea \textsc{assert} tag structure: first pass\label{asserttag:pass1}
\ea	$[$\SB{SAP} $[$\SB{CP} $[$\SB{CP} \textsc{decl} $[$\SB{IP} Lucy is coming $]] [$\SB{C} $\bigwedge ] [$\SB{CP} \textsc{q} isn't $[$\SB{IP} she t\SB{isn't} \sout{coming} $]]]$↘$]$\label{asserttag1:b}
\ex	$[$SAP \textsc{assert}: I am committed\\ 
 \phantom{-SAP-}$[[$\SB{CP1} \textsc{decl}: to the one proposition in the following (singleton) set being true: Lucy is coming]\\
 \phantom{-SAP-}$[$AND$]$\\
 \phantom{-SAP-}$[$\SB{CP2} \textsc{q}: to one of the propositions in the following set being true: Lucy is 
 	 	coming; Lucy is not coming$]]$\label{asserttag1:c}
    \z\z

In other words: the speaker asserts both that Lucy is coming is true, and that either Lucy is coming or Lucy is not coming is true. An \textsc{assert} tag structure essentially is very similar to an asserted declarative -- indeed, it contains one. This suggests that \textsc{assert} tag structures should be interpreted and responded to much like canonical asserted declaratives, so example, a speaker may use an \textsc{assert} tag structure not to elicit a new-to-the-speaker answer from the addressee but to elicit acknowledgement, e.g. because they want to indicate to their interlocutor that they know that their assertion may not be new news, but they still want it to be explicitly part of the discourse content. We already saw examples of acknowledgement-type tags in \xxref{3:know}{3:feet} above.

Turning now to a \textsc{question} tag structure, this receives a rising intonation contour ($\nearrow$) over the tag element. Here, the speaker expects a response from their addressee, e.g. to confirm the proposition in the anchor. We did not see an example of such a use of negative tag structures in the earliest (pre-2;0) examples in our corpus. A first pass structure and paraphrase for a \textsc{question} tag structure is as follows:

\ea \textsc{question} tag structure: first pass\label{questiontag:pass1}
\ea	$[$\SB{SAP} $[$\SB{CP} $[$\SB{CP} \textsc{decl} $[$\SB{IP} Lucy is coming $]] [$\SB{C} $\bigwedge ] [$\SB{CP} \textsc{q} isn't $[$\SB{IP} she t\SB{isn't} \sout{coming} $]]]$↗$]$\label{questiontag1:b}
\ex	$[$SAP \textsc{question}: I am committed to resolving the issue\\ 
 \phantom{-SAP-}$[[$\SB{CP1} \textsc{decl}: of the one proposition in the following set being true: Lucy is coming]\\
 \phantom{-SAP-}$[$AND$]$\\
 \phantom{-SAP-}$[$\SB{CP2} \textsc{q}: of one of the propositions in the following set being true: Lucy is 
 	 	coming; Lucy is not coming$]]$
    \z\z

Through \xref{questiontag:pass1}, we predict that negative tags with rising intonation will be responded to and intended more like canonical information-seeking questions because \xref{questiontag:pass1} contains an interrogative clause scoped over by a \textsc{question} operator.

\xref{asserttag:pass1} and \xref{questiontag:pass1} predict that there may be some sense of redundancy or contradiction associated with the use of tag structures because one proposition from the set that could be true (i.e. from the tag) is the same as the proposition expressed in the anchor. Claims of redundancy in tag structures have, in fact, been made before by linguists (e.g. \citealt{lakoffr1975}), particularly with reference to polarity matched tags (e.g. You're coming, are you?, e.g. \citealt{oconnor1955}), and by non-linguists (e.g. psychiatrists, \citealt{winefieldetal1989}). We will expand on why this apparent redundancy does not result in infelicity in reverse polarity tags in \sectref{sect:clausetypesnotpersps}.

Note once more that conjunction in tag structures by our analysis is at the clausal, not at the speech act level. Speech act conjunction is shown in \xref{4:contradiction} where a declarative is asserted and a polar interrogative containing the same proposition (and its negation) is asked as a question. This creates a clear logical contradiction. 

\ea {\#}Lucy is coming and isn't she?\label{4:contradiction}
\z

(\ref{4:contradiction}) makes clear that we are dealing with a single speech act in the production of a negative tag structure, but that act is not self-evidently a type of question. In fact, much like the Canadian English examples in \xxref{4:canadianrise}{4:canadianlevel}, where declaratives are modified by the discourse particle \textit{eh}, negative tag structures can be either assertions or questions depending on the intonational contour.

\ea You have a new dog, eh?↗\label{4:canadianrise}
\ex All the girls came from the West eh$\rightarrow$, to work in the factory. \label{4:canadianlevel} \\\phantom{a} \jambox*{(Adapted from \citealt[587, 589]{wiltschkoetal2018})}
\z
\il{Canadian English}

In \xref{4:canadianrise}, the rising contour on \textit{eh} contributes to the utterance meaning ``Confirm you have a new dog", which has a question-like use and response pattern. In contrast the level contour on \textit{eh} in \xref{4:canadianlevel} means ``I believe you agree with me", such that this \textit{eh} has an assertion-like use in narratives.

We turn now in more detail to the structure of the right-most adjoined clause; the tag.

\subsubsection{The tag}\label{sect:tagpart}

In \xref{negtag:pass1}, we claim that the tag part of a tag structure is not derived from a NegQ. In terms of existing accounts, our approach is most similar to \citet{holmberg2016}, shown in \xref{holmbergtag}. Note that, according to Holmberg, <+Pol> represents an affirmative declarative variable in C and <±Pol > represents a question variable. Holmberg treats <Pol>, however, as separate from illocutionary force (represented by Q-force). 

\ea	 $[$Q-force $[$\SB{CP} $[$\SB{CP1} Lucy <+Pol> is coming$] [$\SB{CP2} $[$\SB{C} isn't <±Pol>$] [$\SB{PolP} she <±Pol > coming $]]]]$\\\phantom{a} \jambox*{(Based on \citealt[185]{holmberg2016})} \label{holmbergtag}
\z

Note that we diverge from Holmberg in proposing that some negative tag structures (specifically, falling ones) are scoped over by assertive force, whereas the tag structures in \citet{holmberg2016} all carry question force.\footnote{The confirmation-type rising negative tag structure, which we think typically aligns with our \textsc{question} tag structure, was the focus of \citet{holmberg2016}, whereas we look to account for all child negative tag structures in our naturalistic corpus data.}

In proposing \xref{holmbergtag}, Holmberg differentiates negative tags from NegQs. He states that ``they are $[$\ldots$]$ formally different in that the $[$proposition towards which there is bias$]$ is encoded as a clause with valued (positive) polarity in the tag structure, but is derived by application of the high negation to the question variable in the $[$NegQ$]$." \citep[188]{holmberg2016} His formulation of a NegQ is shown in \xref{holmbergnegq}:

\ea $[$\SB{CP} Q-force $[$\SB{CP} Neg $[$\SB{CP} $[$±Pol$]$ $[$\SB{C} $[$PolP …<±Pol>…$]]]]]$ \\\phantom{a} \jambox*{(Based on \citealt[189]{holmberg2016})} \label{holmbergnegq}
\z

Note that negation in \xref{holmbergnegq} scopes above the question variable [$\pm$ Pol] and so does not behave like propositional negation in terms of polarity licensing, amongst other things. We subscribe to this view too, proposing the following structure for NegQs:

\ea\label{negq:pass1}
\ea Isn't Lucy coming?
 \ex $[$\SB{SAP} \textsc{question} $[$\SB{PerspectiveP} Is$+$n't $[$\SB{CP} \textsc{q} t\SB{is} $[$\SB{IP} Lucy t\SB{is} coming $]]]]$
\z\z

The differences between \xref{holmbergnegq} and \xref{negq:pass1} are largely notational and pertain to our different perspectives on the left periphery rather than to differences in the structure of NegQs specifically. 

A question raised here, then, is what interrogative clause typing is doing in NegQs and in negative tags, as in both our and Holmberg's proposals, this is a point of commonality between the two structures.

\paragraph{Separating clause typing from perspectives from speech acts}\label{sect:clausetypesnotpersps}\phantom{a}\\

Clause-typing in English is intimately linked with tense and aspect phenomena; that is, grammaticalized methods for expressing a proposition, situating it in time, and indicating what item in the world it should associate with. Clauses are typed by the realisation and relative position of subjects and elements bearing markers of (non-)finiteness. 

In standard adult English, when an overt subject precedes a tensed verb (auxiliary or lexical), a canonical declarative results and a truth value or set of possible worlds is typically indicated. When a tensed auxiliary verb precedes the overt subject, a canonical (polar) interrogative results and possible truth values or sets of sets of possible worlds are typically indicated. When a non-finite form appears in a root clause without an overt subject, an imperative results, which indicates a property that the speaker wishes were true of the world. 

Though these clause types align canonically with certain perspectives or commitments, this alignment may be disrupted in a number of ways, whether through embedding, discourse particles, polarity operators, intonation, or other means. A non-exhaustive set of examples of canonical and non-canonical uses of clause types are shown in \tabref{tab:clausetypes}.

\begin{table}
\begin{tabularx}{\textwidth}{llQ}
  \lsptoprule
  Clause type & Canonical use & Non-canonical use \\
  \midrule
  Declarative  &  Assertion  &  Question (e.g. rising declaratives); Request (with modal auxiliaries); Exclamative (with intonation); Command (with attitude verbs e.g. \textit{want})  \\
\midrule
  Interrogative  &  Question & Request (e.g. with \textit{please}); Exclamative (e.g. negative polar exclamatives); Assertion (e.g. \textit{fuck}-inversion; \citealt{sailor2020})  \\
\midrule
  Imperative  &  Command  &  Request (e.g. with \textit{please})  \\
   \lspbottomrule
 \end{tabularx}
 \caption{Canonical and non-canonical uses of clause types in English}
\label{tab:clausetypes}
\end{table}

This means that clause type does \textit{not} inherently carry information about perspective or commitment\footnote{See also \citet{schmitz2021} for a similar recent proposal, which runs counter to a Fregean perspective in which sentence `mood' (clause typing) and speech act are often conflated.} -- that is to say, how the speaker uses a clause or intends it to be understood. Cross-linguistic evidence abounds that clause-typing is separate from speaker perspective and commitment, from the mechanisms of discourse particles (e.g. Canadian English \textit{eh} in \xxref{4:canadianrise}{4:canadianlevel}, Romanian \textit{oare} in \citealt{conigzeg2012} and \citealt{farkas2022}, West Flemish \textit{kwestje} in \citealt{woodshaegesubm}) to the interpretation of embedded clauses. Moreover, embedded typed clauses are easily dissociable from their canonical speech acts. For example, interrogative clauses under response verbs like \textit{know} are not interpreted as open questions, but rather as something like the answers to that question (e.g. \citealt{lahiri2002, uegaki2015}).

Returning to the role of clause types in negative tag structures, the conjunction of different clause types in a negative tag structure means that it points both to a set of possible worlds and a set of sets of possible worlds, where the former is a subset of the latter. Though this seems to be redundant, it is important to note there are (at least) two perspectives at play when a negative tag structure is used. 

Let us start with an \textsc{assert} tag structure. If a proposition is simultaneously true and maybe true for the same person, the questioning of the proposition seems redundant, or even contradictory. However, we noted in \sectref{sect:qualdata} and \sectref{sect:negtagpass1} that \textsc{assert} tag structures tend to be used when the speaker recognises that the addressee might already know the proposition to be true. This runs against the core felicity condition of an assertion: that the speaker believes that the addressee does not already believe that some proposition is true (\citealt{searle1969}, \citealt{farkas2022}). However, that proposition might not have been accepted publicly as true by the addressee, whether because they have previously refused to accept it or it has not been addressed directly by the addressee.\footnote{An interesting case of such uses of tag questions with falling intonation is in the case of predicates of personal taste. For an example, imagine that a co-worker mentions that she has a new neighbour, and blushes while saying so. You might say: \textit{He's attractive, isn't he?} to determine the cause of the blushing (see \citealt{malsteph2015}). In this case the speaker is making a guess as to the addressee's point of view, but it is important that it is \textit{presented} as an assertion by the speaker for the addressee to publicly weigh in on. Such cases require more independent work -- thanks to Dan Goodhue for bringing them to our attention.} In either case, \textsc{assert} tag structures are used when the issue of the proposition is not publicly settled for the addressee. Therefore, if the anchor reflects the speaker's commitments, but the tag reflects the speaker's perception that the addressee lacks public commitment to the proposition in this discourse context, there is no longer any redundancy in the structure or its use. We can consider the negative tag structure to carry a conventional implicature (in the sense of \citealt{potts2005}), in that the speaker chooses to assert their proposition using a negative tag structure because they also want to communicate how they view the addressee's commitments.

We therefore update \xref{asserttag:pass1} to reflect this by adding a \textsc{speaker} operator into the PerspectiveP of the declarative clause and an \textsc{addressee} operator into the PerspectiveP of the interrogative clause to represent their differing worlds of evaluation.\footnote{We are agnostic as to the exact mechanisms for how the world of evaluation interacts with material in the CP. There are many approaches that are compatible with our proposal, including \citet{tsoulaskural1999}, \citet{speastenny2003} and \citet{schwarz2012}, i.a. Approaches such as \citet{sigurdsson2014} are not compatible with our proposal as in that case, the speaker/addressee features are lower in the clausal hierarchy than e.g. the clause-typing head.}

\ea \textsc{assert} tag structure: second pass\label{asserttag:pass2}
\ea Lucy is coming, isn't she.↘
\ex$[$\SB{SAP} $[[$\SB{PerspP} \textsc{speaker}  $[$\SB{CP} \textsc{decl} $[$\SB{IP} Lucy is coming $]]] [\bigwedge ] [[$\SB{PerspP} \textsc{addressee} $[$\SB{CP} \textsc{q} isn't $[$\SB{IP} she t\SB{isn't} \sout{coming} $]]]]$↘$]$\label{asserttag2:b}
    \z\z

What about \textsc{question} tag structures? Here we claim again that the two parts of the tag structure relativise to different interlocutors. The felicity conditions of canonical questions include speaker ignorance and addressee competence – i.e. that the speaker doesn't know the answer and the addressee does – as well as addressee compliance – i.e. that the addressee will provide the true answer \citep{farkas2022}. However, there are a number of non-canonical questions in which speaker ignorance is weakened because the speaker may already have evidence for the true answer, e.g. rising declaratives \citep{gunlogson2001}. Given that this type of non-canonical question operationalises a steep rise in intonation, which we also see in \textsc{question} tag structures \citep[140]{dehebraun2013}, it is possible that the same process is active here: the speaker reflects their perception of addressee competence in the anchor\footnote{This analysis is actually quite similar to \citeauthor{malsteph2015}'s (2015) approach to rising tags whose polarity matches that of the anchor, as they claim that such tags place the commitment for the proposition in the addressee's future commitments and not those of the speaker. However, they also mitigate the speaker's commitment to the proposition in their account of mismatching tags, by suggesting that the speaker only commits provisionally to the proposition, on the proviso that the addressee confirms its truth. Moreover, they do not address how the structure of the tag and anchor result in these discourse effects. As we do not address matching tags in this chapter, and \citet{malsteph2015} restrict themselves to rising tags, more work is to be done on how intonation contour and polarity interact.} and their need for addressee compliance (due to their own uncertainty) in the tag. Therefore, \xref{questiontag:pass1} can be updated for \textsc{question} tag structures as follows:

\ea \textsc{question} tag structure: second pass\label{questiontag:pass2}
\ea Lucy is coming, isn't she?↗
\ex$[$\SB{SAP} $[[$\SB{PerspP} \textsc{addressee}  $[$\SB{CP} \textsc{decl} $[$\SB{IP} Lucy is coming $]]] [\bigwedge ] [[$\SB{PerspP} \textsc{speaker} $[$\SB{CP} \textsc{q} isn't $[$\SB{IP} she t\SB{isn't} \sout{coming} $]]]]$↗$]$\label{questiontag2:b}
    \z\z

A difference to note between the proposal here and rising declaratives is that, in the case of rising declaratives, the addressee is publicly known to know the proposition \citep[84--85]{gunlogson2001}. This needn't necessarily hold in the case of a tag structure (see also \citealt{heppot2011}), hence the interrogative tag formally requests that the addressee make public their commitment to the proposition.

These amendments to the proposal may lead a reader to question whether high level speech act operators are necessary when there are also perspectival operators which appear to be doing the same job – in other words, why can't the difference between \textsc{assert} and \textsc{question} tag structures boil down to the \textsc{speaker}/\textsc{addressee} operators and their relationship to the different types of clause?

One reason that \textsc{speaker}/\textsc{addressee} operators and their interaction with clau\-se types cannot alone explain the different structures is intonation. There is evidence from languages like Korean that intonation expresses that a particular response pattern is expected – i.e. that a specific speech act is being performed \citep{ceong2019}. Korean also represents interlocutor perspective and clause typing separately through specific verbal morphology, as shown in \xxref{koreanfall}{koreanrise}, in which the arrows represent final falling or rising intonation (examples from \citealt[12--13]{ceong2017}).

\ea \label{koreanfall}
\gll Meysi-lul          manna-  ss-             ta-             ko-             ↘\\
     Messi-\textsc{acc} meet-   \textsc{past}-  \textsc{decl}-  \textsc{comp}-  \textsc{speaker-commitment}\\
\glt `I said I met Messi.' \phantom{a} \hfill Reinforcing assertion
\z
\il{Korean} %add ``Latin" to language index for this page

\ea \label{koreanrise}
\gll Meysi-lul          manna-  ss-             ta-             ko-            ↗\\
     Messi-\textsc{acc} meet-   \textsc{past}-  \textsc{decl}-  \textsc{comp}-  \textsc{addressee-commitment}\\
\glt `Are you saying you met Messi?' \phantom{a} \hfill Echo question 
\z
\il{Korean} %add ``Latin" to language index for this page

This justifies the inclusion of specific speech act operators that are separate from representations of perspective. Moreover, intonation does not appear to accompany perspectival shifts in the same way. To our knowledge, shifting phenomena such as monstrous indexicals (e.g. in Uyghur, \citealt{Shklovsky2014}) do not trigger specific intonational contours. Additionally, we might expect a completely different intonational contour for \textsc{assert} tag structures than we actually see if \textsc{speaker}/\textsc{ad\-dres\-see} operators were the locus of intonation, given that (we argue) they contain an \textsc{addressee} operator scoping over an interrogative clause, just like a typical information-seeking question might.

Another reason for including \textsc{speaker}/\textsc{addressee} operators as well as speech act operators is that we believe we have evidence that these are just two examples from a large range of operators, including overt items, can be hosted in PerspectiveP in different constructions, and that these other operators are compatible with the intonational contours introduced by \textsc{assert}/\textsc{question} speech act operators. We present this evidence now, as we turn to discuss NegQs in more detail.

\subsection{Metalinguistic negation}\label{sect:metalingneg}

Recall that we presented our NegQ structure in \xref{negq:pass1}, repeated here:

\begin{exe}
\exr{negq:pass1}
\begin{xlist}
\ex Isn't Lucy coming?
 \ex $[$\SB{SAP} \textsc{question} $[$\SB{PerspectiveP} Is$+$n't $[$\SB{CP} \textsc{q} t\SB{is} $[$\SB{IP} Lucy t\SB{is} coming $]]]]$
\end{xlist}
\end{exe}

In \xref{negq:pass1}, the clitic negation \textit{n't} is above the level of the proposition. Evidence for this stems back to \citegen{ladd1981} observation that NPIs are not licensed in NegQs. 

\ea	 Isn't Jane coming \{too/{\#}either\}?
\z

Many linguists before us have proposed that some negation is metalinguistic as it scopes over some object that is bigger than the proposition alone (\citealt{horn1985, horn1989}; \citealt{wood2014}; \citealt{holmberg2016}, a.o.). We join this tradition in claiming that negation in PerspectiveP negates, from the speaker's perspective, that the typical interpretation of a typed clause holds. This is similar to \citeauthor{krifka2015SALT}'s (\citeyear{krifka2015SALT}; see also \citealt{cohenkrifka2014}) concept of denegation of speech acts, whereby negation over a speech act ``prunes $[$the$]$ legal developments" \citep[330]{krifka2015SALT} of some speech act in a discourse; in other words, it prevents certain discourse continuations that would usually stem from canonical use of some clause type from being licit.

How does this fall out? We propose that negation deployed in PerspectiveP indicates that the speaker rejects the interrogativity of the CP – in other words, that they do not believe that alternatives to the proposition are true. This is an indirect, weak method of expressing belief in the truth of the proposition, hence it is not at odds with the question force and corresponding felicity conditions of the NegQ (the speaker's commitment to resolving the issue and their expectation that the addressee will provide the information required). We paraphrase \xref{negq:pass1} as below:

\ea I the speaker am committed to resolving the issue $[$QUESTION$]$ of my not believing that there is a plausible alternative proposition to $p$ $[$NEG-in-PerspectiveP$]$ in the set $p$ or not $p$ $[$CP Q$]$
\z

This means that biased meanings in tag structures cannot fall out from the same mechanism as NegQs because there is no metalinguistic negation in negative tag structures and therefore no rejection of interrogativity in \textsc{question} tag structures. Bias in \textsc{question} tag structures is generated by the interaction of the \textsc{addressee} operator with the proposition: the speaker expresses that they perceive the addressee as knowing a proposition and as asking them to publicly express this commitment. In \textsc{assert} tag structures, bias is created by asserting the proposition in the anchor. As such, the bias in negative tag structures is not homogenous, and in a \textsc{question} tag structure it is indirect -- it is achieved by the speaker looking to resolve the issue of whether they're right to perceive the addressee as believing the proposition to be true. This could be the root of `deferential' readings of some tag structures (as in \citealt{lakoffr1975}), as the speaker appears to be adopting the addressee's perspective in uttering the declarative clause, but this is an example of discourse inference that we expect to be outside the core, syntactically articulated speech act mechanisms we are presenting here.

If we are right so far, we make predictions for both adult and child English in terms of response patterns to NegQs and negative tags. Our approach predicts that adult NegQs should have a response pattern like, but not identical to, canonical questions and unlike assertions. This is because the speaker does not directly assert belief in the positive proposition; this belief is only implied by the speaker's rejection of the interrogativity of CP. \textsc{assert} tag structures will differ from NegQs as acceptable responses to \textsc{assert} tag structures should be almost identical to that of a canonical assertion, given that they contain a declarative clause under an \textsc{assert} operator. \textsc{question} tag structures, finally, are predicted to provoke response patterns like those of canonical questions, even though they are not neutral questions, because they contain an interrogative clause under a \textsc{question} operator. We return to this prediction in \sectref{sect:strengthspatterns} and \sectref{sect:negresponsepatterns}. 

Our approach also predicts that negative tag structures should be acquired earlier and more accurately than NegQs by English-acquiring children. This is because the negation in tag structures is propositional, therefore lower in the syntactic tree, than the metalinguistic negation of NegQs (following logic first propounded by \citealt{rizzi9394} in his Truncation Hypothesis and updated in \citealt{friedmannetal2021}). Metalinguistic negation also scopes over a more complex object (a CP) and requires more complex computation. It is also likely to be harder to acquire because it is realised in the same way phonologically as cliticised propositional negation, so the child must create two categories for the same phonological realisation. We return to this prediction in \sectref{sect:strengthstarget}.

\subsection{Strengths of our account}\label{sect:strengths}

We turn now to showing how our proposal in \sectref{sect:proposal} accounts for both the child and adult data. We will show that our proposal is compatible with aspects of \citeauthor{asherreese2007}'s (2007) discourse-driven account, and many aspects of \citeauthor{holmberg2016}'s (2016) syntactic approach, at least as far as \textsc{question} tag structures are concerned. \sectref{sect:strengthstarget} demonstrates that children are not as target-like in their production of NegQs as in their production of negative tags, with respect to how clitic negation is realised and how they encode bias outside of tag structures. \sectref{sect:strengthspatterns} shows that negative tag structures differ from NegQs in terms of their response patterns, and that differences obtain between the two types of negative tag structure, further supporting our claim that neither is derived from NegQs. \sectref{sect:negresponsepatterns} demonstrates how these response patterns fall out from the structures of the different constructions, using \citeauthor{farkas2022}'s (2022) update of \citeauthor{farkasbruce2010}'s (2010) Table model.

\subsubsection{Child ``high'' negation structures -- are they target-like?}\label{sect:strengthstarget}

We have already seen in \sectref{sect:ourdata} that negative tag structures are used early and with adult-like syntax. \sectref{sect:quantdata} showed that NegQs are produced later, but we will now show that they display more evidence of non-target-like syntax. 

In 306 negative tag structures up to and excluding age 3, there are only 8 errors concerning using the correct auxiliary and two where a full DP subject is used (we do not report here on tense and agreement errors, as we consider these orthogonal). However, in 32 NegQs up to and excluding age 3, 10 of them contain errors concerning the auxiliary (again, excluding errors of tense and agreement), which hosts negation. In fact, they are all errors of auxiliary doubling like in \xref{4:eat}:

\ea Do they don't eat people up? \phantom{a} \hfill	Nina (Suppes), 2;9 \label{4:eat}
\z
\il{US English}

Some of these examples are also plausibly different from NegQs as the bias appears to be towards the negative proposition; an adult-like paraphrase of \xref{4:eat} might be \textit{Do they not eat people up? }In this case, we have even fewer examples of target-like NegQs in child speech despite clear evidence that they are able to conceive of, and try to express, biased meanings in questions with bias towards a positive proposition. For more on auxiliary doubling errors, see \citet{woodsroeper2020}, who connect this type of error directly to early attempts by children to express biased meanings, where the child is biased towards the negative proposition.  

Given the above, our proposal already accounts for the child data more effectively than approaches in which the tag part of a negative tag structure is an elided NegQ, however that is construed. 

\subsubsection{Response patterns}\label{sect:strengthspatterns}
It has been established since \citet{sadock1971,sadock1974} that both negative tag structures and NegQs show evidence of questionhood (e.g. using the \textit{Tell me\ldots} test) and assertionhood (e.g. the \textit{After all\ldots} test; see also \citealt{asherreese2007}). However, as \quotecite{asherreese2007} work suggests, different types of negative tag structures, and NegQs, invite slightly different response patterns. Our proposal also predicts different response patterns and some work on this by \citet{holmberg2016} already exists, so we devote this section to response patterns.

Let's lay out the data. We take our three structures of interest and compare them with canonical assertions and canonical polar questions, as well as a modalised assertion, given that some proposals for tag questions postulate a modal operator in the anchor (e.g. \citealt{billkoevvol} [this volume]). We will use the basic proposition \textit{Bilal is coming}. We also take the following possible responses: polarity particles with matching elided propositions (\textit{Yes he is; no he isn't}) to model polarity-based answers, polarity particles with non-matching elided propositions to model truth-based answers (\textit{Yes he isn't})\footnote{See \citet[Chapter 4.1]{holmberg2016}, \citet{jones1999}, and references therein for more on truth-based or polarity-based answering systems.},  ``agreement indicators'' \textit{right} and \textit{so he is} (as dubbed and investigated by \citealt{holmberg2016}) and silent acceptance, represented by [silence]. Note that we have used \quotecite{holmberg2016} judgments, which we agree with, except in the modalised assertion condition (which he does not discuss).

The data are as follows:

\begin{exe}
\ex \textsc{assert} tag \label{response:asserttag}
   \begin{xlist} 
  \exi{A:} Bilal is coming, isn't he.$\searrow$ 
 \exi{B:} Yes (he is); \#Yes (he isn't); No (he isn't); Right; So he is; [silence].
  \end{xlist} 
\ex  \textsc{question} tag
\begin{xlist} 
  \exi{A:} Bilal is coming, isn't he?$\nearrow$
\exi{B:} Yes (he is); \#Yes (he isn't); No (he isn't); Right; So he is; \#[silence].
\end{xlist} 
\ex NegQ
  \begin{xlist}
  \exi{A:} Isn't Bilal coming? 
\exi{B:} Yes (he is); \%\footnote{Based on author intuitions, this is fine in UK English but not in US English.} Yes (he isn't);  No (he isn't), ??Right; \#So he is; \#[silence].
\end{xlist} 
\ex Assertion
  \begin{xlist}
         \exi{A:}Bilal is coming.
 	\exi{B:}Yes (he is); \#Yes (he isn't); No (he isn't); Right; So he is; [silence].
  \end{xlist} 
\ex Modal assertion
    \begin{xlist}
  \exi{A:}Bilal might be coming.
 	\exi{B:}Yes \#(he is); \#Yes (he isn't)\footnote{While \textit{He isn't} is fine as a response to a modalised assertion, it is the use of \textit{yes} here that leads to infelicity.}; No (he isn't); Right; \#So he is; [silence].
  \end{xlist} 
\ex Polar question \label{response:polar}
      \begin{xlist}
  \exi{A:}Is Bilal coming?
\exi{B:} Yes (he is); \#Yes (he isn't); No (he isn't); \#Right; \#So he is; \#[silence].
\end{xlist} 
\end{exe}
\il{US English}
\il{British English}

This paradigm demonstrates that an \textsc{assert} tag structure patterns just like an assertion in terms of licit responses to it. \textsc{question} tag structures differ in that they require a response – silence is not appropriate – but they differ from canonical polar questions by allowing the agreement indicators (see also \citealt[183]{holmberg2016}). NegQs pattern most closely with canonical polar questions, with some gradability or dialectal variation with respect to truth-based responses and confirmational \textit{right}.

Why should response patterns differ between negative tag structures and Neg\-Qs when existing assertionhood and questionhood tests suggest that they are the same? We think that existing tests are not granular enough when it comes to understanding the nuances around bias and interlocutor perspective in these two structures: in short, speaker belief that the proposition may be true is not the end of the story.

Recall that we spelled out the difference between speaker belief and speaker commitment in our motivation of an extended left periphery in \sectref{sect:proposal}. We have devised a test to demonstrate this distinction with respect to tag structures and NegQs which we call the \textit{Don't you agree\ldots? }test. Using the English verb \textit{agree}, it is possible either to agree with an asserted proposition or with the person who expresses that proposition. These are expressed using different pronouns to point to these different referents. We use agreement with the asserted proposition (\textit{that}) to reflect matching commitments to the proposition in the shared discourse context and agreement with the individual (\textit{me}) to reflect matching perspectives. What we find is that it is possible to ask whether an addressee agrees with the proposition, but not the perspective (i.e. the speaker), following a NegQ \xref{agree:negq} or a \textsc{question} tag structure \xref{agree:qtag}, but with either following an \textsc{assert} tag structure \xref{agree:atag}:

\ea Isn't Jane a good choice? $[$silence from addressee$]$ Don't you agree with that/{\#}me?\label{agree:negq}
\ex Jane's a good choice, isn't she?$\nearrow$ $[$silence from addressee$]$ Don't you agree with that/{\#}me?\label{agree:qtag}
\ex Jane's a good choice, isn't she.$\searrow$ $[$silence from addressee$]$ Don't you agree with that/me?\label{agree:atag}
\z

Moreover, \textit{Don't you agree\ldots} is completely incompatible with canonical polar questions and entirely natural with canonical assertions.

\ea	 Is Jane a good choice? {\#}Don't you agree with that/me?
\ex	 Jane is a good choice. Don't you agree with that/me?
\z

In these examples the pie is cut slightly differently again, but along lines predicted by our analysis: our structures of interest with \textsc{question} operators pattern together, and the structures with \textsc{assert} operators pattern differently. In short, NegQs and \textsc{question} tag structures do not offer up a proposition asserted by the speaker to agree with, despite having some assertion-like properties (see \citealt{sadock1971,sadock1974,asherreese2007}, i.a.). 

This suggests that \textsc{assert} tag structures differ from NegQs in that they express speaker commitment in addition to foregrounding some proposition. This chimes with recent work by \citet{ceong2019}, \citet{krifka2014, krifka2021}, \citet{wiltschkoheim2016, wiltheim2021}, \citet{woods2021please} and \citet{woods2021why} that commitment is a part of grammar over and above (doxastic) belief. It also highlights that longstanding diagnostics for assertion can obscure nuanced differences between different types of non-canonical speech act. 

One last new diagnostic. We borrow and extend a test that \citet[10]{asherreese2007} apply to NegQs to our three structures of interest – the \textit{prior beliefs} test. Note that boldface is used to indicate stress on the auxiliary verbs in T and we include arrows here to indicate intonation contours.

\ea	 I have no prior beliefs on the matter. I just want to know that Lucy \textbf{is} coming, isn't she?$\nearrow$ \label{prior:qtag}
\ex	 I have no prior beliefs on the matter. I just want to know {\#}\textbf{is}n't Lucy coming?
\ex	 I have no prior beliefs on the matter. I just want to know that {\#}Lucy \textbf{is} coming, isn't she.$\searrow$
\z

Another distinction emerges – where a \textsc{question} tag structure is compatible with the speaker claiming no prior beliefs, NegQs and \textsc{assert} tag structures are not. \xref{prior:qtag} in particular supports our claims about \textsc{question} tag structures that the ``assertion flavour'' of them is derived and not directly attributable to the speaker.

We represent the different profiles of our three structures of interest in \tabref{tab:highneg:profiles}, where we mark whether the different types of ``high'' negation structure pattern with canonical assertions, canonical questions, or neither, in our three diagnostics. While \textsc{assert} tag structures are predominantly assertion-like, and \textsc{question} tag structures/NegQs are predominantly question-like, the latter two diverge in different ways and all structures also diverge from both questions and assertions too. All of these nuances are captured by our proposal.

\begin{table}
\begin{tabularx}{\textwidth}{XXXl}
\lsptoprule
       &  \textsc{assert} tag   & \textsc{question} tag & NegQ \\
  \midrule
 Response patterns  &  Assertions  &  Neither & Questions  \\
 \textit{Don't you agree}\ldots  &  Assertions  &  Questions & Questions  \\
 Prior beliefs  &  Neither  &  Questions & Neither  \\
   \lspbottomrule
 \end{tabularx}
 \caption{Negative tag and NegQ profiles}
\label{tab:highneg:profiles}
\end{table}

\subsubsection{Modelling response patterns}\label{sect:negresponsepatterns}
\largerpage
We will now use a model grounded in representing discourse moves to formalise how the propositional and extrapropositional information expressed in negative tag structures and NegQs is communicated by the constituent parts of their structures, resulting in the response patterns mapped in the previous section. To do this, we utilise \quotecite{farkasbruce2010} Table model, updated by \citet{farkas2022}. The model maps how propositions, as part of utterances, move from discourse commitment spaces relativised to interlocutors, into a negotiated and negotiable conversational space known as the Table, and from there into public, shared commitments (cf. previous work by \citealt{stalnaker1979} through \citealt{gunlogson2008}). Along the way, choices made by the speaker about \textit{how} to present these propositions, syntactically and prosodically, communicate something about how they expect the addressee to respond (relative to that proposition). The model can also be used to capture how extra-propositional material affects the movement of the proposition through these spaces.

The full discourse structure outlined above is represented in \tabref{farkasbasic}.

\begin{table}
\begin{tabularx}{\textwidth}{ | Z{80pt} | Z{80pt} | Z{80pt} | Z{80pt} | }
\hline
\textbf{Discourse commitments of speaker (DC\SB{Sp})} & \multicolumn{2}{Z{160pt} |}{\textbf{Table}} & \textbf{Discourse commitments of addressee (DC\SB{Ad})}\\\hline
 & \multicolumn{2}{Z{160pt} |}{ } &  \\\hline
\multicolumn{4}{ | Z{320pt} | }{\textbf{Projected Set (ps)}}\\\hline
\end{tabularx}
\caption{Basic discourse structure \citep[305]{farkas2022}}
\label{farkasbasic}
\end{table}

When an utterance is pronounced, the propositional content is placed onto the Table. Broader informational content, which includes, for example, the speaker's commitment with respect to the proposition, is placed into the speaker's discourse commitments. The projected set is then generated; this consists of adding to the addressee's discourse commitments some proposition, such that that proposition would constitute a canonical response to the utterance if the addressee were to commit to it. If there is more than one proposition for which this is the case, then the projected set is not a singleton set. 

\tabref{farkasdecl} demonstrates this process when an assertive utterance containing a declarative clause is uttered. The sincere utterer of such an utterance commits\footnote{Explicitly spelling \textit{commitment} out in the speaker's discourse commitments is technically redundant, as the fact of the speaker's placing \textit{p} on the Table indicates their public commitment to \textit{p}. However, we want to be fully explicit for clarity and to highlight the contrasting perspectives that the speaker manipulates.} to a single true proposition \textit{p}. They place \textit{p} on the Table, and project a single future discourse move, namely that the addressee will commit to \textit{p} too. If the addressee is cooperative, they will demonstrate commitment to \textit{p}, though this may be implicit, as acceptance of (via commitment to) \textit{p} can be considered a default response, as it's the only response indicated by the speaker's utterance choices. The presence of \textit{p} on the Table and in the addressee public discourse commitments leads to redundancy, so \textit{p} can then be added to the interlocutors' shared discourse commitments and be considered a resolved issue.

\begin{table}
\begin{tabularx}{\textwidth}{ | Z{80pt} | Z{80pt} | Z{80pt} | Z{80pt} | }
\hline
\textbf{DC\SB{Sp}} & \multicolumn{2}{Z{160pt} |}{\textbf{Table}} & \textbf{DC\SB{Ad}}\\\hline
Sp commits to $p$ & \multicolumn{2}{Z{160pt} |}{$\{p\}$} &  \\\hline
\multicolumn{4}{ | Z{320pt} | }{\textbf{ps:} $\{DC\SB{Ad}\cup\{p\}\} $}\\\hline
\end{tabularx}
\caption{Conversational state following the utterances of a declarative with propositional content $p$ \citep[308]{farkas2022}}
\label{farkasdecl}
\end{table}

Polar interrogative utterances differ in that the speaker commits not to propositions, but to worlds in which holds a set consisting of two propositions, \textit{p} or not \textit{p }– in other words, all possible worlds that are compatible with the current discourse context. We can rephrase this in commitment terms as a commitment by the speaker that the question of \textit{p} or not \textit{p} is open and unresolved in the current discourse context, and this is added to their public discourse commitments. The set $\{\textit{p},\neg\textit{p}\}$ is added to the Table to be resolved. The projected set consists of the addressee committing to either \textit{p} (e.g. by responding \textit{yes}) or not \textit{p} (by responding \textit{no}). This is modelled in \tabref{farkasq}. Note that a response such as \textit{I don't know} is not canonical because in a canonical information-seeking question, the utterer of the question should believe that the addressee knows and can provide the true answer (recall mention of addressee competence in our discussion of \textsc{question} tag structures in \sectref{sect:clausetypesnotpersps}). Moreover, an explicit response is required in cases like \tabref{farkasq} because there is no single projected set and hence no default addressee response.

\begin{table}
\begin{tabularx}{\textwidth}{ | Z{80pt} | Z{80pt} | Z{80pt} | Z{80pt} | }
\hline
\textbf{DC\SB{Sp}} & \multicolumn{2}{Z{160pt} |}{\textbf{Table}} & \textbf{DC\SB{Ad}}\\\hline
Sp commits to wanting to resolve $\{p, \neg p\}$ & \multicolumn{2}{Z{160pt} |}{$\{p,{\neg}p \}$} &  \\\hline
\multicolumn{4}{ | Z{320pt} | }{\textbf{ps:} $\{DC\SB{Ad}\cup\{p\},DC\SB{Ad}\cup\{{\neg}p\}\} $}\\\hline
\end{tabularx}
\caption{Conversational state following the utterance of a polar interrogative querying $p$ (adapted from \citealt[312]{farkas2022})}
\label{farkasq}
\end{table}

Let's now see how our three structures of interest play out in the Table model. We argued in \sectref{sect:negtagpass1} and \sectref{sect:clausetypesnotpersps} that the \textsc{assert} tag structure is fundamentally an assertion of \textit{p} whose structure conventionally implicates the speaker's belief in a second not-at-issue proposition \textit{q}, namely that the addressee has not yet committed to \textit{p} publicly.\footnote{Incidentally \textsc{assert} tag structures and their proposed meaning are a good test case for pure intensionalist vs. commitment-based models of discourse exchanges (see \citealt{geurts2019} for more on the matter). It is very complex to express the addressee's lack of public commitment to \textit{p} in terms of knowing or believing, because lack of public commitment need not be due to lack of knowledge/belief. \textsc{assert} tag structures are compatible both with the addressee knowing or not knowing \textit{p}, but crucially are only licensed when the addressee has not already \textit{publicly committed }to \textit{p}, as explained in \sectref{sect:clausetypesnotpersps}.} The speaker commitments, both at-issue and not-at-issue, are expressed in the speaker's discourse commitments (DC\SB{Sp}) while the at-issue propositional material is expressed on the Table. Given that the \textsc{assert} tag structure is fundamentally an assertion, and because the speaker is committing explicitly to \textit{p} (this is indicated by the intonation contour), the projected set of discourse moves is a singleton set in which the addressee also publicly commits to \textit{p}, just like in the canonical assertion in \tabref{farkasdecl}. This is consistent with the overarching discourse aim of clearing the Table because if the addressee commits to \textit{p}, both sets of propositions on the Table are resolved; it is also compatible with the individual interlocutors' public discourse commitments. This is schematised in \tabref{farkasasserttag}.

\begin{table}
\begin{tabularx}{\textwidth}{ | Z{80pt} | Z{80pt} | Z{80pt} | Z{80pt} | }
\hline
\textbf{DC\SB{Sp}} & \multicolumn{2}{Z{160pt} |}{\textbf{Table}} & \textbf{DC\SB{Ad}}\\
\hline
Sp commits to $p$; Sp commits to $q$ (= Ad hasn't yet publicly committed to $p$)
 & \multicolumn{2}{Z{160pt} |}{<Lucy is coming> = $\{p\}$; <Isn't she coming> = $\{p,{\neg}p \}$} &  \\
 \hline
\multicolumn{4}{ | Z{320pt} | }{\textbf{ps:} $\{DC\SB{Ad}\cup\{p\}\} $}\\
\hline
\end{tabularx}
\caption{Conversational state an utterance of \textsc{assert} tag structure \textit{Lucy is coming, isn't she.}}
\label{farkasasserttag}
\end{table}

Note that to disconfirm \textit{p} in \tabref{farkasasserttag} is considered a non-canonical move, just as it is following a canonical assertion. This doesn't mean that disconfirming \textit{p} is impossible, but rather that it will take some negotiation between speaker and addressee until they agree, and \textit{p} is resolved into their shared commitments, or until they agree to disagree.

In contrast, in a \textsc{question} tag structure, the speaker commits to wanting to resolve the question of \{\textit{p}, $\neg$\textit{p}\}. They also express a not-at-issue proposition $q$ about the beliefs of the addressee, namely that the addressee believes \textit{p}. The Table is exactly the same as in the \textsc{assert} tag structure in \tabref{farkasasserttag} but the projected set reflects the speaker's at-issue public commitments as indicated by the intonation contour – the speaker expects the addressee to commit to \textit{p} or not \textit{p} and either would be considered a canonical response. Note that because the projected set is not a singleton set, silence cannot be used as default method of committing to \textit{p}. This discourse impact of uttering a \textsc{question} tag structure is schematized in \tabref{farkasquestiontag}.


\begin{table}
\begin{tabularx}{\textwidth}{ | Z{80pt} | Z{80pt} | Z{80pt} | Z{80pt} | }
\hline
\textbf{DC\SB{Sp}} & \multicolumn{2}{Z{160pt} |}{\textbf{Table}} & \textbf{DC\SB{Ad}}\\
\hline
Sp commits to wanting to resolve $\{p, \neg p\}$; Sp commits to $q$  (= Ad believes $p$) & \multicolumn{2}{Z{160pt} |}{<Lucy is coming> = $\{p\}$; <Isn't she coming> = $\{p,{\neg}p \}$} &  \\
\hline
\multicolumn{4}{ | Z{320pt} | }{\textbf{ps:} $\{DC\SB{Ad}\cup\{p\},DC\SB{Ad}\cup\{{\neg}p\}\} $}\\
\hline
\end{tabularx}
\caption{Conversational state an utterance of \textsc{question} tag structure \textit{Lucy is coming, isn't she?}}
\label{farkasquestiontag}
\end{table}

A brief note on not-at-issueness: the propositions \textit{q} in \tabref{farkasasserttag} and \tabref{farkasquestiontag} are not-at-issue because they cannot be directly challenged, though they can be demonstrated to be wrong:

\begin{exe}
\ex Intended reading: challenging A's expression that B lacks public commitment
\begin{xlist}
\ex{A.} Lucy is coming, isn't she.
\ex{B.} No, \#I already said that.		
\ex{B′.} Yeah, I already said that.
  
\end{xlist}
\end{exe}

\begin{exe}
\ex Intended reading: challenging A's expression that B believes $p$
\begin{xlist}
\ex{A.} Lucy is coming, isn't she?
 \ex{B.} (\#No,) I don't know if she's coming. 	   	     
 \ex{B′.} Sorry, I don't know if she's coming.
\end{xlist}
\end{exe}
 
Moreover, while the speaker in \tabref{farkasquestiontag} might have to repeal their commitment to \textit{q} in the case that the addressee commits to $\neg$\textit{p}, this is still compatible with projecting a set in which a commitment to $\neg$\textit{p} is a canonical discourse move for the addressee because (a) it is compatible with the Table and (b) it is compatible with the speaker's \textit{own} commitment to \textit{p} (because there isn't one). For comparison, in \xref{muirenglishnai}, A presents Laura Muir's nationality as a not-at-issue proposition in an appositive relative clause (see \citealt{potts2005}). In response, B can respond canonically to the question and separately point out A's mistake (as in \xref{canon1} and \xref{canon2}), but cannot directly challenge A's mistake using a polarity particle see \xref{noncanon} in the way that they might if Laura Muir's nationality were presented as at-issue content (see \xref{muirenglishai}).

\begin{exe}
    \ex \textit{Laura Muir is English} = not-at-issue\label{muirenglishnai}
    \begin{xlist}
  \ex{A.} Did you hear that Laura Muir, that amazing English runner, won silver in Tokyo?
  \ex{B.} I did but, er, Laura Muir is Scottish.\label{canon1}
  \ex{B′.} No, but, er, Laura Muir is Scottish.\label{canon2}
  \ex{B′′.} \#No, Laura Muir is Scottish.\label{noncanon}
    \end{xlist}
\end{exe}

\begin{exe}
\ex \textit{Laura Muir is English} = at-issue \label{muirenglishai}
    \begin{xlist}
\exi{A.} Laura Muir is English.
\exi{B.} No, she's Scottish.
    \end{xlist}
\end{exe}

Finally, we must account for the fact that one can respond using agreement indicators like \textit{right} and \textit{so she is} to a \textsc{question} tag structure. We assume that these indicators can target the singleton proposition \textit{p} on the table, which simultaneously ``counts" as committing to the member \textit{p} of the non-singleton set \{\textit{p}, $\neg$\textit{p}\}. Recall that the singleton proposition \textit{p} is not on the Table as a single item in canonical polar questions, hence the difference in response patterns.

In (brief) summary, \textsc{question} tag structures are like \textsc{assert} tag structures in terms of the propositions that are at issue (i.e. on the Table), but they differ in terms of projected sets. They are similar in that speakers of both make at-issue commitments with respect to \textit{p} and make not-at-issue commitments about the addressee's stance on \textit{p}, though the exact nature of these commitments differs. This accounts for the differences we find in response patterns but the similarities in acquisition: they contain identical amounts and types of at-issue material, identical syntactic structures, and equally complex speaker commitments.

Now we turn to NegQs. NegQs place one set of at-issue propositions onto the Table. Like with the negative tag structures above, this non-singleton set contains \textit{p} and not \textit{p}. As the negation in a NegQ is analysed as metalinguistic, the Table for a NegQ looks just like the Table for a canonical polar question and the speaker commits to wanting to resolve \{\textit{p}, $\neg$\textit{p}\}. The projected set is then predicted to be identical to that of a canonical polar question too. However, the speaker commitments expressed by a NegQ are not equivalent to those in neutral polar questions. We argued above that metalinguistic negation essentially negates the plausibility of the alternative to \textit{p} (see also \citealt[188]{holmberg2016}). If we model this as in \tabref{farkasnegqfail}, then the projected set must be a singleton set DC\SB{Ad}$\cup$\{\textit{p}\}, because DC\SB{Ad}$\cup$\{$\neg$\textit{p}\} would be incompatible with the speaker's public discourse commitments. Ultimately, then, the model fails, because there is no projected set that follows from both the speaker's public discourse commitments.


\begin{table}
\begin{tabularx}{\textwidth}{ | Z{80pt} | Z{80pt} | Z{80pt} | Z{80pt} | }
\hline
\textbf{DC\SB{Sp}} & \multicolumn{2}{Z{160pt} |}{\textbf{Table}} & \textbf{DC\SB{Ad}}\\
\hline
Sp commits to wanting to resolve $\{p, \neg{p}\};\neg\{\neg{p}\}=p$
 & \multicolumn{2}{Z{160pt} |}{<Is Lucy coming> = $\{p,{\neg}p \}$} &  \\
 \hline
\multicolumn{4}{ | Z{320pt} | }{\textbf{ps:} ?}\\
\hline
\end{tabularx}
\caption{A failed model for NegQs if metalinguistic negation = propositional (logical) negation}
\label{farkasnegqfail}
\end{table}

The problem is the representation of the negation in \tabref{farkasnegqfail}. Recall that we argued in the tradition of \citet{horn1985,horn1989} that metalinguistic negation is not formally equivalent to propositional (logical) negation. If it were, the wrong predictions (or no predictions) about response patterns would be made, as they are in \tabref{farkasnegqfail}. We see that if the speaker commits to $\neg\{\neg{p}\}$, this reduces to \textit{p} and it should not be projected that a canonical move for the addressee is to commit to $\neg$\textit{p}, but we know that it can be. 

To avoid the failure of \tabref{farkasnegqfail}, we represent metalinguistic negation using all caps (NOT) in \tabref{farkasnegq}.

\begin{table}
\begin{tabularx}{\textwidth}{ | Z{80pt} | Z{80pt} | Z{80pt} | Z{80pt} | }
\hline
\textbf{DC\SB{Sp}} & \multicolumn{2}{Z{160pt} |}{\textbf{Table}} & \textbf{DC\SB{Ad}}\\
\hline
Sp commits to wanting to resolve $\{p, \neg{p}\}$; Sp commits to $q$ 
(= NOT $\{\neg{p}\}$)
 & \multicolumn{2}{Z{160pt} |}{<Is Lucy coming> = $\{p,{\neg}p \}$} &  \\
\hline
\multicolumn{4}{ | Z{320pt} | }{\textbf{ps:} $\{DC\SB{Ad}\cup\{p\},DC\SB{Ad}\cup\{{\neg}p\}\} $ }\\
\hline
\end{tabularx}
\caption{Conversational state after an utterance of NegQ ``Isn't Lucy coming?"}
\label{farkasnegq}
\end{table}

The entry in the speaker's discourse commitments in \tabref{farkasnegq} means that the speaker commits to there being no plausible alternative to \textit{p} given their knowledge and beliefs. This is not logically equivalent to asserting \textit{p} because the truth of \textit{p} could be left undefined, therefore it is only implied that the speaker must, therefore, believe \textit{p} to be true. In Horn-style terms, the speaker is registering their objections to accepting $\neg$\textit{p} in the face of some possible evidence for it. This follows traditional accounts of metalinguistic negation that claim that ``rectification or correction is a necessary part of the interpretation of $[$\ldots$]$ metalinguistic negation" (\citealt[689]{kay2004}, discussing \citealt{horn1985}). However, because NegQs fold metalinguistic negation into a question structure, the burden of rectifying or correcting falls on the addressee rather than on the user of metalinguistic negation themselves.

Because the speaker does not formally commit to \textit{p}, the inclusion of DC$_{Ad}\cup\{\neg{p}\}$ as a canonical response to the NegQ is formally licit and does not logically contradict the speaker's public commitments.

If we are correct, the analyses above demonstrate more precisely why negative tag structures are acquired before NegQs. Negative tag structures are transparent in that their at-issue content directly follows from their surface structure. NegQs, on the other hand, are not transparent as their at-issue content is ``less than'' the phonologically expressed material. To put this another way, metalinguistic negation is phonologically expressed in the middle of (indeed, it is cliticised to) propositional, at-issue material, but is not itself propositional or at-issue. It is therefore a complex task for the child to separate out at-issue and not-at-issue expressions that are phonologically tightly bound together.


\section{Summary}\label{summary}

In this chapter we focused on the production of English nuclear negative tag structures and NegQs by very young children and adults, as well as their contribution to discourse. 

Our empirical contributions are as follows. We created a dataset containing over 600 utterances of ``high'' negation structures by 67 English-acquiring children, demonstrating that negative tag structures precede and outnumber NegQs. Qualitatively, the dataset shows that children use negative tag structures accurately both in terms of adult-like syntax and discourse contribution. We also discussed a particular non-target-like NegQ that uses auxiliary doubling, noting that target-like NegQs with positive bias are even more rare than the dataset suggests.

We also demonstrated using adult judgements that negative tag structures in English divide into two types, both of which are also distinct from NegQs, in terms of interpretation and in terms of response pattern. We propose two new diagnostics, the \textit{don't you agree?} test and the \textit{prior beliefs }test that further refine our understanding of speaker commitment and belief in the deployment of negative tag structures and NegQs.

Our empirical findings feed our theoretical claims. Contra much of the existing literature (see \citealt{holmberg2016} and \cite{krifka2015SALT} for exceptions), we argue that English nuclear negative tag structures are simple speech acts that are complex at the clausal level. They do not consist of an assertion combined with a NegQ -- indeed, to speak of complex speech acts creates problems further down the theoretical line in terms of predicting and understanding responses to such acts. Negative tag structures are a declarative clause conjoined with an interrogative clause containing propositional negation, and this whole is interpreted in one of two ways depending on (a) the perspective attributed to each clause (\textsc{speaker} or \textsc{addressee}) and (b) the speech act operator that scopes over the whole (\textsc{assert} or \textsc{question}).

In contrast, English NegQs consist of an interrogative clause scoped over by metalinguistic negation and a \textsc{question} speech act operator. We demonstrated how clauses, perspectives and speech act operators interact using \citeauthor{farkas2022}'s (2022) version of \citeauthor{farkasbruce2010}'s (2010) Table model. Bias is created in nuclear negative tag questions via the relationship of speech act operator, perspective and proposition in the anchor, whereas in NegQs it arises from the metalinguistic rejection of the interrogativity of the CP – in other words, the speaker expresses that they believe there to be no plausible alternative to the main proposition.

\largerpage
A number of areas for future study remain, most pertinently the prosody of tag structures in child speech and in their input. We hope, however, that the breadth of the predictions made by the proposals here, and the empirical evidence that we have been able to offer in this chapter, provide support for the enterprise of trying to understand how children acquire different types of speech acts and will invite energetic debate.

\section*{Abbreviations}

\begin{tabularx} {.45\textwidth}{lQ}
\textsc{acc}   &   accusative \\
Ad              &  addressee \\
\textsc{comp}   &  complementiser \\
DC             &   discourse commitments \\
\textsc{decl}  &   declarative \\
NegQ   &           interrogative \\
\end{tabularx}
\begin{tabularx} {.54\textwidth}{ll}
Q   &              ``high'' negation question \\
PerspP   &         Perspective Phrase \\
Pol(P)   &         Polarity (Phrase) \\
ps   &             projected set \\
SA(P)   &          Speech Act (Phrase) \\
Sp   &             speaker \\
 \end{tabularx}


\section*{Acknowledgements}
We would like to thank: the participants at Leibniz-ZAS's Biased Question workshop in February 2021 (not least Robin Connelly for his \textit{lack} of participation at 5 days old); the participants of the North East Syntax Seminar (May 2022, Newcastle), especially Anders Holmberg and George Tsoulas; colleagues in the online meetings of the UMass-based Language Acquisition Research Center; two anonymous reviewers for their careful and helpful comments; Cory Bill, Dan Goodhue, Johannes Heim and Martina Wiltschko for informal discussion; and Tue Trinh and Kazuko Yatsushiro for their (very patient!) editorship. 

\section*{Guide to CHILDES corpora references and DOIs, in order of first mention}
Manchester corpus: \citet{theakstonetal2001}. DOI: \url{https://doi.org/10.21415/T54G6D}\\ 
Valian corpus: \citet{valian1991}. DOI: \url{https://doi.org/10.21415/T5ZS3T} \\
MacWhinney corpus: \citet{macwhin1991}. DOI: \url{https://doi.org/10.21415/T5JP4F}\\   
Gleason corpus: \citet{masurberko1980}. DOI: \url{https://doi.org/10.21415/T5101R}\\ 
Belfast corpus: \citet{henry1995}. DOI: \url{https://doi.org/10.21415/T5VG79}  \\
Brown corpus: \citet{brown1973}. DOI: \url{https://doi.org/10.21415/T5HK5G} \\
Kuczaj corpus: \citet{kuczaj1977}. DOI: \url{https://doi.org/10.21415/T5H30R} \\
Higginson corpus: \citet{higg1985}. DOI: \url{https://doi.org/10.21415/T5S31M}\\ 
Suppes corpus: \citet{suppes1974}. DOI: \url{https://doi.org/10.21415/T5WS4K} \\
Tardif corpus: DOI: \url{https://doi.org/10.21415/T5TK58}  \\
Wells corpus: \citet{wells1981}. DOI: \url{https://doi.org/10.21415/T5T60K}


\sloppy
\printbibliography[heading=subbibliography,notkeyword=this]
\end{document}
